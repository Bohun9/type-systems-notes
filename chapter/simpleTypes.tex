\chapter{Simple Types}

%%%%%%%%%%%%%%%%%%%%%%%%%%%%%%%%%%%%%%%%%%%%%%%%%%%%%%%%%%%%%%%%%%%%%%%%%%%%%%%
\section{Simply Typed Lambda Calculus}

\newcommand\CaseTerm[3]{\texttt{case}\;#1\;\texttt{of}\;\iota_1\;x.#2\texttt{|}\iota_2\;x.#3}
\newcommand\iotaval{\iota^{\textsc{v}}}

\begin{alignat*}{2}
  \tau & \Coloneqq \tau \to \tau \mid b
    \tag{types} \\
  e & \Coloneqq v \mid e\;e
    \tag{expressions} \\
  v & \Coloneqq x \mid \lambda x.e \mid c_b
    \tag{values}
\end{alignat*}

$\Gamma\vdash e : \tau$ - In context $\Gamma$ expression $e$ has type $\tau$

$\Gamma ::= \emptyset \mid \Gamma, x : \tau$

$(x:\tau) \Leftrightarrow \Gamma(x) = \tau$

$\Gamma, x:\tau \Leftrightarrow \Gamma[x \mapsto \tau]$

\begin{mathpar}
  \inferrule{ }
            {\Gamma\vdash c_b : b}

  \inferrule{(x : \tau) \in \Gamma}
            {\Gamma\vdash x : \tau}
    
  \inferrule{\Gamma, x : \tau_1 \vdash e : \tau_2}
            {\Gamma\vdash \lambda x.e : \tau_1 \to \tau_2}
  
  \inferrule{\Gamma\vdash e_1 : \tau_2 \to \tau_1 \\ \Gamma\vdash e_2 : \tau_2}
            {\Gamma\vdash e_1\;e_2 : \tau_1}
\end{mathpar}

%%%%%%%%%%%%%%%%%%%%%%%%%%%%%%%%%%%%%%%%%%%%%%%%%%%%%%%%%%%%%%%%%%%%%%%%%%%%%%%
\section{Curry-Howard Isomorphism}

Now, let's remove all terms from typing rules above, replace $\times$ with $\land$,
$+$ with $\lor$ and squint our eyes a little. What do we see? A wild natural deduction appears!

\begin{mathpar}
  \inferrule{\tau \in \Gamma}
            {\Gamma\vdash \tau}
  
  \\\\

  \inferrule{\Gamma, \tau_1 \vdash \tau_2}
            {\Gamma\vdash \tau_1 \to \tau_2}
  
  \inferrule{\Gamma\vdash \tau_2 \to \tau_1 \\ \Gamma\vdash \tau_2}
            {\Gamma\vdash \tau_1}

  \\\\
          
  \inferrule{\Gamma\vdash \tau_1 \\ \Gamma\vdash \tau_2}
            {\Gamma\vdash \tau_1 \land \tau_2}

  \inferrule{\Gamma\vdash \tau_1 \land \tau_2}
            {\Gamma\vdash \tau_i}

  \\\\

  \inferrule{\Gamma\vdash \tau_i}
            {\Gamma\vdash \tau_1 \lor \tau_2}

  \inferrule{\Gamma\vdash \tau_1 \lor \tau_2 \\ \Gamma, \tau_i \vdash \tau}
            {\Gamma\vdash \tau}
\end{mathpar}

We can also add product type (tuple) and sum type (coproduct / union) to our calculus:

\begin{alignat*}{2}
  \tau & \Coloneqq \ldots \mid \tau\times\tau \mid \tau+\tau
    \tag{types} \\
  e & \Coloneqq \ldots \mid (e,e)
      \mid \pi_1\;e \mid \pi_2\;e
      \mid \iota_1\;e \mid \iota_2\;e \mid \CaseTerm{e}{e}{e}
    \tag{expressions} \\
  v & \Coloneqq \ldots \mid \langle v, v \rangle
    \mid \iotaval_1\;v \mid \iotaval_2\;v
    \tag{values}
\end{alignat*}

Typing rules for (co)product type:

\begin{mathpar}
  \inferrule{\Gamma\vdash e_1 : \tau_1 \\ \Gamma\vdash e_2 : \tau_2}
            {\Gamma\vdash (e_1, e_2) : \tau_1\times\tau_2}

  \inferrule{\Gamma\vdash e : \tau_1\times\tau_2}
            {\Gamma\vdash \pi_i\;e : \tau_i}
            (i=1,2)

  \\\\

  \inferrule{\Gamma\vdash e : \tau_i}
            {\Gamma\vdash \iota_i\;e : \tau_1+\tau_2}
            (i=1,2)

  \inferrule{\Gamma\vdash e : \tau_1+\tau_2 \\ \Gamma, x : \tau_i \vdash e_i : \tau\ (i=1,2)}
            {\Gamma\vdash \CaseTerm{e}{e_1}{e_2} : \tau}
\end{mathpar}

Maybe we should add also typing rules for runtime constructs
$\langle v_1, v_2 \rangle$ and $\iotaval_i v$.

We then need to extend reduction semantics with the following rules:

\begin{mathpar}
  \inferrule{ }{(v_1, v_2) \rightharpoonup \langle v_1, v_2 \rangle}

  \inferrule{ }{\pi_i\;\langle v_1, v_2 \rangle \rightharpoonup v_i}

  \\\\

  \inferrule{ }{\iota_i\;v \rightharpoonup \iotaval_i\;v}
  
  \inferrule{ }{\CaseTerm{\iotaval_i\;v}{e_1}{e_2} \rightharpoonup e_i\{v/x\}}
\end{mathpar}

and evaluation context:

\begin{alignat*}{2}
  E & \Coloneqq \ldots \mid (E,e) \mid (v,E)
    \mid \pi_1\;E \mid \pi_2\;E
    \mid \iota_1\;E \mid \iota_2\;E 
    \mid \CaseTerm{E}{e_1}{e_2}
\end{alignat*}

%%%%%%%%%%%%%%%%%%%%%%%%%%%%%%%%%%%%%%%%%%%%%%%%%%%%%%%%%%%%%%%%%%%%%%%%%%%%%%%
\section{Progress and Preservation}

Type-safety: If $\emptyset\vdash e: \tau$ then there is no non-value $e'$
such that $e \longrightarrow^* e' \not\longrightarrow \cdot$

XX-century technique: progress + type preservation

Progress: $\emptyset\vdash e: \tau$ then $\exists v.e=v$ or $\exists e'.e \longrightarrow e'$

Progress proof for simply typed lambda calculus - induction on type derivation structure
(exercise for the reader) (will likely require Inversion lemma)

Inversion lemma:
If $\emptyset\vdash v : \tau_1 \to \tau_2$ then $v = \lambda x.e$ for some x and e
(Proof is left as an exercise for the reader)

Type Preservation
$\emptyset\vdash e : \tau$ and $e \longrightarrow e'$ then $\emptyset\vdash e' : \tau$
(Proof is left as an exercise for the reader)

Substitution lemma:
If $\Gamma, x:\tau_1 \vdash e : \tau_2$ and $\Gamma\vdash v : \tau_1$
then $\Gamma\vdash e\{v/x\} : \tau_2$
(Proof is left as an exercise for the reader - weakening lemma might be useful)

Weakening lemma:
If $\Gamma\vdash e : \tau$ then $\Gamma, x:\tau' \vdash e : \tau$ where $x\not\in dom(\Gamma)$
(Proof is left as an exercise for the reader)

%%%%%%%%%%%%%%%%%%%%%%%%%%%%%%%%%%%%%%%%%%%%%%%%%%%%%%%%%%%%%%%%%%%%%%%%%%%%%%%
\section{Further Reading}
