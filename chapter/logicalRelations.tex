\chapter{Logical Relations}

%%%%%%%%%%%%%%%%%%%%%%%%%%%%%%%%%%%%%%%%%%%%%%%%%%%%%%%%%%%%%%%%%%%%%%%%%%%%%%%
\section{Syntactic Provability and Semantic Truth}

Recall the formulae of propositional logic, as described by the following
grammar. For simplicity, we omit variables and consider just the true
and false constants and implication.
\begin{alignat*}{2}
  \varphi & \Coloneqq \top \mid \bot \mid \varphi \to \varphi
\end{alignat*}
We say that a formula is \emph{provable} and write $\vdash \varphi$
when we can derive its proof using natural deduction. Such a derivation
is purely syntactic, and makes no appeal to the meaning of the formula.
On the other hand, to talk about whether a formula is \emph{true},
we must do so in reference to a semantics, such as the one given by the
denotation
$\Denot{\cdot} \colon \syncatset{Formula} \to \mathcal{D}$
defined below\footnote{
  As intuitionists, we can think of $\mathcal{D}$ as a Heyting algebra,
  with $\mathsf{T}$ and $\mathsf{F}$ as its top and bottom elements, and
  $\Rightarrow$ as its implication operator.
}.
\begin{alignat*}{2}
  \Denot{\top} & = \mathsf{T} \\
  \Denot{\bot} & = \mathsf{F} \\
  \Denot{\varphi \to \psi} & = \Denot{\varphi} \Rightarrow \Denot{\psi}
\end{alignat*}
If $\varphi$ is true in accordance to the semantics
(\emph{i.e.}, its denotation is $\mathsf{T}$),
we write $\models \varphi$. How do
provability and truth relate to each other? For one, provability should be
\emph{sound} with respect to the semantic truth, \emph{i.e.}, all provable
formulae are true (${\vdash} \subseteq {\models}$).
In the example we are considering at present, we also have the
converse property, known as \emph{completeness}: true formulae are provable.
Completeness is not always feasible in general, but an incomplete
yet sound system of deduction can still be useful in practice.

Returning to our study of programming languages, we have seen that there is a
correspondence between the typing relation $\Gamma \vdash e : \tau$ and natural
deduction.
Accordingly, we can define a semantic relation $\Gamma \models e : \tau$, in
analogy to $\models \varphi$, such that soundness holds.
Such a relation is usually called a \emph{logical relation}. In the rest of
this chapter, we will define a logical relation for the simply-typed lambda
calculus, give a proof of soundness, and show that the logical relation can be
used to establish various interesting properties, such as type safety (again)
and the termination of well-typed programs.

%%%%%%%%%%%%%%%%%%%%%%%%%%%%%%%%%%%%%%%%%%%%%%%%%%%%%%%%%%%%%%%%%%%%%%%%%%%%%%%
\section{Denotation of Types}

As with the formulae of propositional logic, we need to assign some sort of
semantic interpretation to the syntactic types of the lambda calculus. What is
an appropriate semantic domain for the denotation of types? When we try to
think of what a type represents, for example as we are writing a program, we
tend to think of the set of (closed) values that inhabit that type. Therefore,
we define the space of \emph{semantic types} as
\[
  \semcatset{Type} = \mathcal{P}(\syncatset{Value}_{\varnothing}),
\]
where $\syncatset{Value}_{\varnothing}$ is the set of all closed values.

We will now define the denotation
$\Denot{\cdot} \colon \syncatset{Type} \to \semcatset{Type}$.
We know that $\texttt{()}$ is the only value of type $\texttt{Unit}$,
so we can take the singleton $\{\texttt{()}\}$ as the semantic
interpretation of this type.
Arrow types are a bit more complicated. They represent functions, and the
essence of a function is that it can be applied to an argument. We will exploit
this intuition in the denotation of arrows by saying that a value behaves like
an arrow if, when applied to a semantically well-typed value, the result is
also semantically well-typed. However, there is one problem: an application is
not a value, so it cannot be in the denotation of the result type.
We somehow need to expand the denotation to all expressions.
For this purpose, we define the expression closure operator
$\mathcal{E} \colon \semcatset{Type} \to
  \mathcal{P}(\syncatset{Expr}_{\varnothing})$.

Expressions which are not values can be reduced.
Thus, given a semantic type $R$, the closure $\mathcal{E} R$
should at the very least contain all the expression that reduce
to a value in $R$. What of expressions that do not terminate? Since
type safety permitted non-termination, we might want to do the same in
$\mathcal{E}$.

\begin{figure}[t!!]
\begin{alignat*}{2}
  \Denot{\texttt{Unit}} & = \{\texttt{()}\} \\
  \Denot{\tau_1 \to \tau_2} & = \{ v \mid
    \forall v' \in \Denot{\tau_1}\ldotp v\;v' \in \mathcal{E}\Denot{\tau_2} \} \\\\
  \mathcal{E}R & = \{ e \mid \forall e'\ldotp e \longrightarrow^* e'
  \Rightarrow (\exists v \in R \ldotp e'=v) \lor
    (\exists e'' \ldotp e'\longrightarrow^*e'') \}
\end{alignat*}
\caption{The denotation of types and the expression closure operator.}
\label{fig:logicalRelations_denots}
\end{figure}

As an aside, there is a certain asymmetry in the denotations for
$\texttt{Unit}$ and $\tau_1\to\tau_2$. The former corresponds to the
introduction rule for type $\texttt{Unit}$ and uses the unit value constructor
$\texttt{()}$, while the latter corresponds to the elimination rule for
arrows and uses application.
In our case, we could equivalently define the denotation of an arrow as
\[
  \Denot{\tau_1 \to \tau_2} = \{ \lambda x\ldotp e \mid
  \forall v \in \Denot{\tau_1}\ldotp
    \Subst{e}{v}{x} \in \mathcal{E}\Denot{\tau_2} \}.
\]
This version is similar to the introduction rule for the arrow type, and
requires values of the shape $\lambda x\ldotp e$. The problem with this
formulation is that it is less abstract, and therefore, fails to account for
extensions to the calculus, such as recursive functions. Similarly, we could
try to define a denotation for $\texttt{Unit}$ in the elimination style.
In a sense, this is a degenerate case, as there are no elimination rules for
$\texttt{Unit}$. Therefore, we could place no additional conditions on the
values in the denotation and set
$\Denot{\texttt{Unit}} = \syncatset{Value}_{\varnothing}$. Although
unintuitive, this definition would not compromise any of the theorems
that we will see in this chapter.

%%%%%%%%%%%%%%%%%%%%%%%%%%%%%%%%%%%%%%%%%%%%%%%%%%%%%%%%%%%%%%%%%%%%%%%%%%%%%%%
\section{The Logical Relation}

% From denotation of types to open expressions

$\Denot{\Gamma} = \{\gamma \mid \forall x\ldotp \gamma(x) \in \Denot{\Gamma(x)}\}$
($\gamma \in \Denot{\Gamma} \Longleftrightarrow \forall x\ldotp \gamma(x) \in \Denot{\Gamma(x)} $)

$\gamma \colon X \to \syncatset{Value}_{\varnothing}$

$\Gamma \models e \;:\; \tau \Longleftrightarrow
  \forall \gamma \in \Denot{\Gamma}\ldotp
  \gamma^*(e) \in \mathcal{E}\Denot{\tau}$

%%%%%%%%%%%%%%%%%%%%%%%%%%%%%%%%%%%%%%%%%%%%%%%%%%%%%%%%%%%%%%%%%%%%%%%%%%%%%%%
\section{Soundness}

\begin{theorem}[Fundamental Property]
  If $\Gamma \vdash e \;:\; \tau$ then $\Gamma \models e \;:\; \tau$.
\end{theorem}

\begin{theorem}[Adequacy]
  For any $e \in \mathcal{E}R$ we have $\Safe{e}$.
\end{theorem}

\begin{theorem}[Type safety]
  If $\emptyset \vdash e : \tau$ then $\Safe{e}$.
\end{theorem}

\begin{lemma}\label{lem:lr-val-in-eclo}
  For any semantic type $R$ and value $v \in R$
  we have $v \in \mathcal{E}R$.
\end{lemma}

\begin{lemma}\label{lem:lr-eclo-red}
  If $e \longrightarrow e'$ and $e' \in \mathcal{E}R$ then $e \in \mathcal{E}R$.
\end{lemma}

In order to prove Fundamental Property
we show series of \emph{compatibility lemmas}.
Each of them corresponds to one typing rule.

\begin{lemma}
  $\Gamma \models () \;:\; \mathtt{Unit}$
\end{lemma}

\begin{lemma}
  If $(x:\tau) \in \Gamma$ then
  $\Gamma \models x \;:\; \tau$.
\end{lemma}

\begin{lemma}
  If $\Gamma, x:\tau_1 \models e \;:\; \tau_2$
  then $\Gamma \models \lambda x.e \;:\; \tau_1 \to \tau_2$.
\end{lemma}
\begin{proof}
  Take any $\gamma \in \Denot{\Gamma}$;
    to show: $\gamma^{*}\lambda x.e
      = \lambda x.\gamma^{*} e
      \in \mathcal{E}\Denot{\tau_1 \to \tau_2}$.\\
  By \autoref{lem:lr-val-in-eclo} it suffices to show 
    $\lambda x.\gamma^{*} e
      \in \Denot{\tau_1 \to \tau_2}$.\\
  Take any $v \in \Denot{\tau_1}$;
    to show: $(\lambda x.\gamma^{*} e)\;v
      \in \mathcal{E}\Denot{\tau_2}$. \\
  By \autoref{lem:lr-eclo-red} it suffices to show
    $(\gamma^{*} e)\{v/x\}
    = \gamma[x\mapsto v]^* e \in \mathcal{E}\Denot{\tau_2}$. \\
  By assumption, it suffices to show
    $\gamma[x\mapsto v] \in \Denot{\Gamma, x : \tau_1}$,
  which is easy to verify by case analysis on variable.
\end{proof}

For convenience, we prove compatibility lemma for application
for closed terms first.

\begin{lemma}\label{lem:lr-compat-app-cl}
  If $e_1 \in \mathcal{E}\Denot{\tau_2 \to \tau_1}$
  and $e_2 \in \mathcal{E}\Denot{\tau_2}$
  then $e_1\;e_2 \in \mathcal{E}\Denot{\tau_1}$.
\end{lemma}
\begin{proof}
  Pick any $e'$ such that $e_1\;e_2 \longrightarrow^* e'$.
  We have three, the following cases.
  \begin{enumerate}[label=(\roman*)]
  \item $e_1 \longrightarrow^* e_1'$ (not a value) and $e' = e_1'\;e_2$.
    By assumption we have $e_1' \longrightarrow e_1''$
    and therefore, $e' \longrightarrow e_1''\;e_2$.
  \item $e_1 \longrightarrow^* v_1$, $e_2\longrightarrow e_2'$
    (not a value) and $e' = v_1\;e_2'$.
    By assumption we have $e_2' \longrightarrow e_2''$
    and therefore, $e' \longrightarrow v_1\;e_2''$.
  \item $e_1 \longrightarrow^* v_1$, $e_2 \longrightarrow^* v_2$
    and $v_1\;v_2 \longrightarrow^* e'$.
    By assumption we have $v_1\in\Denot{\tau_2\to\tau_1}$
    and $v_2\in\Denot{\tau_2}$,
    so by the definition of $\Denot{\tau_2\to\tau_1}$
    we have $v_1\;v_2 \in \mathcal{E}\Denot{\tau_1}$.
    Since $v_1\;v_2 \longrightarrow^* e'$ we can conclude the proof.
    \qedhere
  \end{enumerate}
\end{proof}

\begin{lemma}
  If $\Gamma \models e_1 \;:\; \tau_2 \to \tau_1 $
  and $\Gamma \models e_2 \;:\; \tau_2$
  then $\Gamma \models e_1\;e_2 \;:\; \tau_1$.
\end{lemma}
\begin{proof}
  Directly from \autoref{lem:lr-compat-app-cl}.
\end{proof}

\medskip

Using Compatibility Lemmas proof of Fundamental Property is straightforward induction.

%%%%%%%%%%%%%%%%%%%%%%%%%%%%%%%%%%%%%%%%%%%%%%%%%%%%%%%%%%%%%%%%%%%%%%%%%%%%%%%
\section{Termination of Simply-Typed Lambda Calculus}

\begin{theorem}{Termination}
  If $\emptyset \vdash e \;:\; \tau$ then
  there exists $v$ such that $e \longrightarrow^* v$.
\end{theorem}

To prove termination we will change definition of $\mathcal{E}$ to reflect that fact.

$\mathcal{E}R = \{ e \mid \exists v \in R\ldotp e \longrightarrow^* v \}$

%%%%%%%%%%%%%%%%%%%%%%%%%%%%%%%%%%%%%%%%%%%%%%%%%%%%%%%%%%%%%%%%%%%%%%%%%%%%%%%
\section{Further Reading}
