\chapter{Syntax}

%%%%%%%%%%%%%%%%%%%%%%%%%%%%%%%%%%%%%%%%%%%%%%%%%%%%%%%%%%%%%%%%%%%%%%%%%%%%%%%
\section{Grammars and Abstract Syntax Trees}

%%%%%%%%%%%%%%%%%%%%%%%%%%%%%%%%%%%%%%%%%%%%%%%%%%%%%%%%%%%%%%%%%%%%%%%%%%%%%%%
\section{The Classic Approach to Variable Binding}

\newcommand\Preterm{\mathrm{Preterm}}
\newcommand\Var{\mathrm{Var}}
\newcommand\Subst[3]{#1\{#2/#3\}}

$\Preterm$ is smallest set, such that:

\begin{itemize}
\item $\forall x \in \Var\ldotp x \in \Preterm$
\item $\forall x \in \Var\ldotp\forall t \in \Preterm\ldotp \lambda x.t \in \Preterm$
\item $\forall t_1, t_2 \in \Preterm\ldotp t_1\;t_2 \in \Preterm$
\end{itemize}

Probem: some distinct preterms such as $\lambda x \ldotp x$ and $\lambda y \ldotp y$
differ in bound variable names, but represent the same computation.

Solution: $\alpha$-equivalence relation - terms that differ only in the
names of bound variables will belong to the same equivalence class.

But first, substitution on preterms (partial function):

\[
  \Subst{\cdot}{\cdot}{\cdot} \colon
  \Preterm \times \Var \times \Preterm \to \Preterm
\]
\begin{eqnarray*}
  \Subst{x}{t}{x}          & = & t \\
  \Subst{y}{t}{x}          & = & y, \text{if } x \ne y \\
  \Subst{(t_1\;t_2)}{t}{x} & = & (\Subst{t_1}{t}{x})\;(\Subst{t_1}{t}{x}) \\
  \Subst{(\lambda x.t')}{t}{x} & = & \lambda x.t' \\
  \Subst{(\lambda y.t')}{t}{x} & = & \lambda y.\Subst{t'}{t}{x}, \text{if } x \ne y \land y \not\in FV(t)
\end{eqnarray*}

$\alpha$-equivalence relation:

\newcommand\Aeq{\equiv_\alpha}

\begin{mathpar}
  \inferrule{ }
            {t \Aeq t}

  \inferrule{t_1 \Aeq t_2 \\ t_2 \Aeq t_3}
            {t_1 \Aeq t_3}

  \inferrule{t_2 \Aeq t_1}
            {t_1 \Aeq t_2}

  \inferrule{ }
            {x \Aeq x}

  \inferrule{t_1 \Aeq t_1' \\ t_2 \Aeq t_2'}
            {t_1\;t_2 \Aeq t_1'\;t_2'}

  \inferrule{t_1 \Aeq t_2}
            {\lambda x.t_1 \Aeq \lambda x.t_2}

  \inferrule{y \not\in FV(t)}
            {\lambda x.t \Aeq \lambda y.\Subst{t}{y}{x}}

\end{mathpar}

%%%%%%%%%%%%%%%%%%%%%%%%%%%%%%%%%%%%%%%%%%%%%%%%%%%%%%%%%%%%%%%%%%%%%%%%%%%%%%%
\section{Variable Binding via Indexed Families of Sets}

% Indexed family of sets of terms
\newcommand\ITerm[1]{\mathrm{Term}_{#1}}
% Variable constructor
\newcommand\ITVar[1]{\ulcorner #1 \urcorner}
% Successor (constructor of X+1)$
\newcommand\ITSucc{\mathsf{s}}
% Extending set by one element
\newcommand\ITInc[1]{#1\!+\!1}

Another approach to construct terms is to define a set $\ITerm{X}$,
where $X$ is a set of variables that are potentially free.

For example $\ITerm{\text{VAR}}$ would contain all terms and
$\ITerm{\emptyset}$ -- all closed terms (without any free variables).

Formally $\ITerm{X}$ is smallest set, such that:

\begin{itemize}
\item $\forall x \in X\ldotp \ITVar{x} \in \ITerm{X}$
\item $\forall t \in \ITerm{X+1}\ldotp \lambda t \in \ITerm{X}$
\item $\forall t_1, t_2 \in \ITerm{X}\ldotp t_1\;t_2 \in \ITerm{X}$
\end{itemize}

$\ITInc{X}$ is isomorphic to \texttt{Maybe} or \texttt{option}
and is defined as:

\begin{itemize}
\item $0 \in \ITInc{X}$
\item $\forall x \in X\ldotp \ITSucc\;x \in \ITInc{X}$
\end{itemize}

We should note that we changed $\lambda x.t$ to simply $\lambda t$.
The role of $x$ now belongs to 0, which may be present in $t \in \ITerm{X+1}$.
For example term $\lambda x.\lambda y. xy$
would be written as $\lambda\lambda \ITSucc\;0\;0 = \lambda\lambda 1\;0$.

With the removal of names for bound variables we don't run into the same problems as before.
The only thing left to do is to define substitution
\[
	\cdot\{\cdot\}:
  \ITerm{X+1} \times \ITerm{X} \to \ITerm{X}
\]

and to do that we need to define:

$^* \colon (X \to \ITerm{Y}) \to \ITerm{X} \to \ITerm{Y}$ (also known as \texttt{bind})

\begin{eqnarray*}
  f^*\ITVar{x}   & = & f\;x \\
  f^*(\lambda t) & = & \lambda (f^{\Uparrow*} \; t) \\
  f^*(t_1\;t_2)  & = & (f^* \; t_1) \; (f^* \; t_2)
\end{eqnarray*}

$^\Uparrow \colon (X \to \ITerm{Y}) \to \ITInc{X} \to \ITerm{\ITInc{Y}}$

\begin{eqnarray*}
  f^\Uparrow 0 & = & \ITVar{0} \\
  f^\Uparrow (\ITSucc \; x) & = & \ITSucc^\dagger\;(f\;x) \\
\end{eqnarray*}

$^\dagger \colon (X \to Y) \to \ITerm{X} \to \ITerm{Y}$ (also known as \texttt{fmap})

\begin{eqnarray*}
  f^\dagger\ITVar{x}   & = & \ITVar{f\;x} \\
  f^\dagger(\lambda t) & = & \lambda(f^{\uparrow\dagger}\;t) \\
  f^\dagger(t_1\;t_2)  & = & (f^\dagger\;t_1)\;(f^\dagger\;t_2)
\end{eqnarray*}

$\uparrow \colon (X \to Y) \to \ITInc{X} \to \ITInc{Y}$
  (also known as \texttt{fmap} but on \texttt{Maybe})

\begin{eqnarray*}
  f^\uparrow 0            & = & 0 \\
  f^\uparrow (\ITSucc\;x) & = & \ITSucc\;(f \; x)
\end{eqnarray*}

%%%%%%%%%%%%%%%%%%%%%%%%%%%%%%%%%%%%%%%%%%%%%%%%%%%%%%%%%%%%%%%%%%%%%%%%%%%%%%%
\section{Further Reading}
