\chapter{Biorthogonal Logical Relations}

In the last chapter we quietly skipped over this fact, but properties such
as type safety or termination are really properties of complete programs,
rather than of individual expressions. For the vanilla simply-typed lambda
calculus, this distinction is not crucial. However, as we have seen in
\autoref{sec:non-local-red}, once we add control operators such as
\texttt{call/cc}, reduction must be defined for complete programs, since we
need knowledge of the surrounding context.

This has a profound effect on our treatment of logical relations so far.
Whereas previously we could rely on properties of reduction that talked
about the reduction of subexpressions, such as $e_1 \longrightarrow^* e_1'$
implying $e_1\;e_2 \longrightarrow^* e_1'\;e_2$, this is no longer the case in
the presence of non-local control flow. Consider the simple program
\[
  (\mathcal{K} k\ldotp \lambda x\ldotp k\;x)\;(\lambda x\ldotp \texttt{()}).
\]
Clearly, we should not attempt to perform any reduction on the subexpression
$(\mathcal{K} k\ldotp \lambda x\ldotp k\;x)$ alone, since then we would
reify the incorrect context. Thus, the reasoning we used in the proofs of
\autoref{lem:lr-compat-app-cl} and \autoref{lem:lr-compat-app-cl-sn} is not
only inelegant, but also invalid for more sophisticated calculi.
This also suggests that our previous attempts at defining the expression
closure operator will not do in this case, since we demanded \emph{all}
expressions to fulfill the global program properties of type safety and
termination.

%%%%%%%%%%%%%%%%%%%%%%%%%%%%%%%%%%%%%%%%%%%%%%%%%%%%%%%%%%%%%%%%%%%%%%%%%%%%%%%
\section{Type System for \texttt{call/cc}}

\begin{mathpar}
    \inferrule{\Gamma,x : (\tau \to \bot) \vdash e : \tau}
              {\Gamma\vdash \mathcal{K} x.e : \tau}    
\end{mathpar}

%%%%%%%%%%%%%%%%%%%%%%%%%%%%%%%%%%%%%%%%%%%%%%%%%%%%%%%%%%%%%%%%%%%%%%%%%%%%%%%
\section{Biorthogonal Closure}


$\mathcal{E}R = \{ e \mid \forall E \in \mathcal{K} R. E[e] \in \Obs \}$

$\mathcal{K}R = \{ E \mid \forall v \in R. E[v] \in \Obs \}$

Type safety: $\Obs = \rawSafe$

Termination: $\Obs = \{ p \mid \exists v. p \longrightarrow^* v\}$

\begin{lemma}
  If $e_1 \in \mathcal{E}\Denot{\tau_2 \to \tau_1}$
  and $e_2 \in \mathcal{E}\Denot{\tau_2}$
  then $e_1\;e_2 \in \mathcal{E}\Denot{\tau_1}$.
\end{lemma}
\begin{proof}
  Take any $E \in \mathcal{K}\Denot{\tau_1}$;
    to show: $E[e_1\;e_2] = (E\;e_2)[e_1] \in \Obs$. \\
  By assumption ($e_1 \in \mathcal{E}\Denot{\tau_2 \to \tau_1}$),
    it suffices to show $E\;e_2 \in \mathcal{K}\Denot{\tau_2 \to \tau_1}$. \\
  Take any $v_1 \in \Denot{\tau_2 \to \tau_1}$;
    to show: $(E\;e_2)[v_1] = E[v_1\;e_2] = (v_1\;E)[e_2] \in \Obs$. \\
  By assumption ($e_2 \in \mathcal{E}\Denot{\tau_2}$),
    it suffices to show $v_1\;E\in\mathcal{K}\Denot{\tau_2}$. \\
  Take any $v_2 \in \Denot{\tau_2}$;
    to show: $(v_1\;E)\;v_2 = E[v_1\;v_2] \in \Obs$, which holds
    by the fact that $v_1\in\Denot{\tau_2\to\tau_1}$
    and assumptions over $v_2$ and $E$.
\end{proof}

\begin{lemma}\label{lem:biortho_obs_red}.
  If $p \longrightarrow p'$ and $p' \in \Obs$, then $p \in \Obs.$
\end{lemma}

\begin{lemma}
  If $\Gamma,x : \tau_1 \models e : \tau_2$
  then $\Gamma\models \lambda x.e : \tau_1 \to \tau_2$.
\end{lemma}
\begin{proof}
  Take any $\gamma \in \Denot{\Gamma}$;
    to show: $\gamma^*(\lambda x.e) =
      \lambda x.\gamma^* e \in \Denot{\tau_1 \to \tau_2}$. \\
  Take any $v \in \Denot{\tau_1}$;
    to show: $(\lambda x.\gamma^* e)\;v \in \mathcal{E}\Denot{\tau_2}$.\\
  Take any $E \in \mathcal{K}\Denot{t_2}$;
    to show: $E[(\lambda x.\gamma^* e)\;v] \in \Obs$.\\
  By \autoref{lem:biortho_obs_red} it suffices to show
    $E[(\gamma^* e)\{v/x\}] = E[\gamma[x\mapsto v]^* e] \in \Obs$. \\
  By assumption ($\Gamma,x : \tau_1\models e : \tau_2$)
    it suffices to show $\gamma[x\mapsto v] \in \Denot{\Gamma,x : \tau_1}$,
    which can be easily verified by a simple case analysis.
\end{proof}

%%%%%%%%%%%%%%%%%%%%%%%%%%%%%%%%%%%%%%%%%%%%%%%%%%%%%%%%%%%%%%%%%%%%%%%%%%%%%%%
\section{Further Reading}
