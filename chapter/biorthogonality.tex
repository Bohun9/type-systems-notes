\chapter{Biorthogonal Logical Relations}

In the last chapter we quietly skipped over this fact, but properties such
as type safety or termination are really properties of complete programs,
rather than of individual expressions. For the vanilla simply-typed lambda
calculus, this distinction is not crucial. However, as we have seen in
\autoref{sec:non-local-red}, once we add control operators such as
\texttt{call/cc}, reduction must be defined for complete programs, since we
need knowledge of the surrounding context.

This has a profound effect on our treatment of logical relations so far.
Whereas previously we could rely on properties of reduction that talked
about the reduction of subexpressions, such as $e_1 \longrightarrow^* e_1'$
implying $e_1\;e_2 \longrightarrow^* e_1'\;e_2$, this is no longer the case in
the presence of non-local control flow. Consider the simple program
\[
  (\mathcal{K} k\ldotp \lambda x\ldotp k\;x)\;(\lambda x\ldotp \texttt{()}).
\]
Clearly, we should not attempt to perform any reduction on the subexpression
$(\mathcal{K} k\ldotp \lambda x\ldotp k\;x)$ alone, since then we would
reify the incorrect context. Thus, the reasoning we used in the proofs of
\autoref{lem:lr-compat-app-cl} and \autoref{lem:lr-compat-app-cl-sn} is not
only inelegant, but also invalid for more sophisticated calculi.
This also suggests that our previous attempts at defining the expression
closure operator will not do in this case, since we demanded \emph{all}
expressions to fulfill the global program properties of type safety and
termination.

%%%%%%%%%%%%%%%%%%%%%%%%%%%%%%%%%%%%%%%%%%%%%%%%%%%%%%%%%%%%%%%%%%%%%%%%%%%%%%%
\section{Type System for \texttt{call/cc}}

In this chapter we extend our simple lambda calculus from
\autoref{ch:simple-types} with \texttt{call/cc} operator
\[
  \mathcal{K} k\ldotp e
\]
that lets us formazlize the behaviour of throwing and catching an exception.
Because we now differentiate between whole programs and subexpressions
we need to define a new grammar for the language.
\begin{alignat*}{2}
  \syncatset{Expr} & \ni e && \Coloneqq v \mid e\;e \mid \texttt{absurd}\;e
    \mid \mathcal{K} x\ldotp e \mid \mathcal{A}p
    \tag{expressions} \\
  \syncatset{Prog} & \ni p && \Coloneqq e
    \tag{programs} \\
  \syncatset{Val} & \ni v && \Coloneqq x \mid \lambda x.e \mid c_b
    \tag{values} \\
  \syncatset{Ectx} & \ni E && \Coloneqq \square \mid E \; e \mid v \; E
    \tag{evaluation contexts}
\end{alignat*}
and we extend $\syncatset{Type}$ with $\bot$ constant.
Besides the \texttt{call/cc} operator we added two more language constructs
\begin{itemize}
  \item $\texttt{absurd}$, that is just a regular $\bot$-elimination
    function
  \item $\mathcal{A}p$ (\textit{abort}), which replaces currently evaluated
    program with $p$ - its is required by operational semantics and it
    should never be used in hand-written programs
\end{itemize}

We can think of evaluation contexts as of "functions" transforming
expressions into whole programs, i.e. "$E : e \rightarrow p$". To make this
intuition formal we define \textit{plug} operation in obvious way.
What the reader should keep in mind is that the hole in an evaluation context
is supposed to be at the very position where the evaluation happens.
It is alseo worth to notice that evaluation contexts cannot contain
\texttt{call/cc} or \texttt{abort}. This enforces the correct order of
computation where the whole \texttt{call/cc} (or \texttt{abort}) expression
is always evaluated before any of its subexpressions.

We define following reduction rules
\begin{alignat*}{2}
  E[(\lambda x.e)\;v] & \longrightarrow E[e\{v / x\}] \\
  E[\mathcal{K} x.e]  & \longrightarrow E[e\{\lambda y.\mathcal{A} E[y] / x\}] \\
  E[\mathcal{A} p]    & \longrightarrow p
\end{alignat*}
First rule is standard. Third rule defines behaviour of \texttt{abort}
like it was described above - whole evaluation context is replaced by $p$.
Second rule describes \texttt{call/cc}. It says that when we evaluate this
construct it turns into just the expression it wrapped, but with all
occurrences of the bound variable replaced by a function, which, applied
to any value and evaluated, reverts the state of the computation to
the point where we started computing related \texttt{call/cc} and
resumes from that point assuming the given argument as the result.
This behaviour is analogous to what we experience from \texttt{try-catch}
blocks - the bound variable $x$ in $\mathcal{K} x.e$ corresponds to a function
\texttt{raise} or \texttt{throw} that interrupts computation and raises
a specialized exception (with a single parameter) that is only caught by
the specialized block (i.e. the one binding $x$ - $x$ acts here as a kind
of exception type). The only difference is that in regular programming
languages we also have a special exception-handling block, while here
computation is immediately resumed with the thrown value, but it is just
a technical detail that can be easily overcome.

Let us see an example program with \texttt{call/cc}.
\begin{alignat*}{2}
  \mathcal{K} \textit{zero} \ldotp \texttt{List.fold\_left}\;(\lambda & p\;x\ldotp \\
  & \textbf{if} \; x = 0 \; \textbf{then} \; \textit{zero}\;0 \\
  & \textbf{else} \; p*x)
\end{alignat*}
This expression, in a context where it is applied to a list, takes a list
and computes to a product of its elements. In case all of them are nonzero
it just performs a fold, but if one of them is zero this expression takes
a shortcut by immediately reducing to $0$.

Now, when we understand the meaning of \texttt{call/cc}, we can derive a typing
rule for it. There are two obvious observations: first, the type of the whole
construct should be the type of the expression it wraps, and second, the variable
$x$ bound by \texttt{call/cc} should be a function accepting values of this
type as arguments. By an analogy to \texttt{raise} we want to be able to put
$(x\;\textit{someparameter})$ everywhere, but we do not want to add polymorphism
to our system. Here is where the $\bot$ type comes in handy as it can be
converted to any type by the \texttt{absurd} rule.
\begin{mathpar}
    \inferrule{\Gamma\vdash e : \bot}
              {\Gamma\vdash \texttt{absurd}\;e : \tau}
\end{mathpar}
That is why our rule has the following shape:
\begin{mathpar}
    \inferrule{\Gamma,x : (\tau \to \bot) \vdash e : \tau}
              {\Gamma\vdash \mathcal{K} x.e : \tau}
\end{mathpar}
As a final remark let us notice that, after erasing terms (and defining $\neg\tau$
as a syntactic sugar for $\tau\rightarrow\bot$), the rule turns to
\begin{mathpar}
    \inferrule{\Gamma, \neg\tau \vdash \tau}
              {\Gamma\vdash \tau}
\end{mathpar}
and that is exactly Pierce's law. This means that if we extend our lambda
calculus with products, sums and $\top$ (i.e. \texttt{unit}) we can
encode classical logic, natural deduction proofs as $\lambda$-terms.

%%%%%%%%%%%%%%%%%%%%%%%%%%%%%%%%%%%%%%%%%%%%%%%%%%%%%%%%%%%%%%%%%%%%%%%%%%%%%%%
\section{Biorthogonal Closure}
\label{sec:biorthogonal-closure}


$\mathcal{E}R = \{ e \mid \forall E \in \mathcal{K} R. E[e] \in \Obs \}$

$\mathcal{K}R = \{ E \mid \forall v \in R. E[v] \in \Obs \}$

Type safety: $\Obs = \rawSafe$

Termination: $\Obs = \{ p \mid \exists v. p \longrightarrow^* v\}$

\begin{lemma}
  If $e_1 \in \mathcal{E}\Denot{\tau_2 \to \tau_1}$
  and $e_2 \in \mathcal{E}\Denot{\tau_2}$
  then $e_1\;e_2 \in \mathcal{E}\Denot{\tau_1}$.
\end{lemma}
\begin{proof}
  Take any $E \in \mathcal{K}\Denot{\tau_1}$;
    to show: $E[e_1\;e_2] = (E\;e_2)[e_1] \in \Obs$. \\
  By assumption ($e_1 \in \mathcal{E}\Denot{\tau_2 \to \tau_1}$),
    it suffices to show $E\;e_2 \in \mathcal{K}\Denot{\tau_2 \to \tau_1}$. \\
  Take any $v_1 \in \Denot{\tau_2 \to \tau_1}$;
    to show: $(E\;e_2)[v_1] = E[v_1\;e_2] = (v_1\;E)[e_2] \in \Obs$. \\
  By assumption ($e_2 \in \mathcal{E}\Denot{\tau_2}$),
    it suffices to show $v_1\;E\in\mathcal{K}\Denot{\tau_2}$. \\
  Take any $v_2 \in \Denot{\tau_2}$;
    to show: $(v_1\;E)\;v_2 = E[v_1\;v_2] \in \Obs$, which holds
    by the fact that $v_1\in\Denot{\tau_2\to\tau_1}$
    and assumptions over $v_2$ and $E$.
\end{proof}

\begin{lemma}\label{lem:biortho_obs_red}.
  If $p \longrightarrow p'$ and $p' \in \Obs$, then $p \in \Obs.$
\end{lemma}

\begin{lemma}
  If $\Gamma,x : \tau_1 \models e : \tau_2$
  then $\Gamma\models \lambda x.e : \tau_1 \to \tau_2$.
\end{lemma}
\begin{proof}
  Take any $\gamma \in \Denot{\Gamma}$;
    to show: $\gamma^*(\lambda x.e) =
      \lambda x.\gamma^* e \in \Denot{\tau_1 \to \tau_2}$. \\
  Take any $v \in \Denot{\tau_1}$;
    to show: $(\lambda x.\gamma^* e)\;v \in \mathcal{E}\Denot{\tau_2}$.\\
  Take any $E \in \mathcal{K}\Denot{t_2}$;
    to show: $E[(\lambda x.\gamma^* e)\;v] \in \Obs$.\\
  By \autoref{lem:biortho_obs_red} it suffices to show
    $E[(\gamma^* e)\{v/x\}] = E[\gamma[x\mapsto v]^* e] \in \Obs$. \\
  By assumption ($\Gamma,x : \tau_1\models e : \tau_2$)
    it suffices to show $\gamma[x\mapsto v] \in \Denot{\Gamma,x : \tau_1}$,
    which can be easily verified by a simple case analysis.
\end{proof}

%%%%%%%%%%%%%%%%%%%%%%%%%%%%%%%%%%%%%%%%%%%%%%%%%%%%%%%%%%%%%%%%%%%%%%%%%%%%%%%
\section{Further Reading}

