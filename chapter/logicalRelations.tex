\chapter{Logical Relations}

%%%%%%%%%%%%%%%%%%%%%%%%%%%%%%%%%%%%%%%%%%%%%%%%%%%%%%%%%%%%%%%%%%%%%%%%%%%%%%%
\section{Syntactic Provability and Semantic Truth}

Recall the formulae of propositional logic, as described by the following
grammar. For simplicity, we omit variables and consider just the true
and false constants and implication.
\begin{alignat*}{2}
  \varphi & \Coloneqq \top \mid \bot \mid \varphi \to \varphi
\end{alignat*}
We say that a formula is \emph{provable} and write $\vdash \varphi$
when we can derive its proof using natural deduction. Such a derivation
is purely syntactic, and makes no appeal to the meaning of the formula.
On the other hand, to talk about whether a formula is \emph{true},
we must do so in reference to a semantics, such as the one given by the
denotation
$\Denot{\cdot} \colon \syncatset{Formula} \to \mathcal{D}$
defined below\footnote{
  As intuitionists, we can think of $\mathcal{D}$ as a Heyting algebra,
  with $\mathsf{T}$ and $\mathsf{F}$ as its top and bottom elements, and
  $\Rightarrow$ as its implication operator.
}.
\begin{alignat*}{2}
  \Denot{\top} & = \mathsf{T} \\
  \Denot{\bot} & = \mathsf{F} \\
  \Denot{\varphi \to \psi} & = \Denot{\varphi} \Rightarrow \Denot{\psi}
\end{alignat*}
If $\varphi$ is true in accordance to the semantics
(\emph{i.e.}, its denotation is $\mathsf{T}$),
we write $\models \varphi$. How do
provability and truth relate to each other? For one, provability should be
\emph{sound} with respect to the semantic truth, \emph{i.e.}, all provable
formulae are true (${\vdash} \subseteq {\models}$).
In the example we are considering at present, we also have the
converse property, known as \emph{completeness}: true formulae are provable.
Completeness is not always feasible in general, but an incomplete
yet sound system of deduction can still be useful in practice.

Returning to our study of programming languages, we have seen that there is a
correspondence between the typing relation $\Gamma \vdash e : \tau$ and natural
deduction.
Accordingly, we can define a semantic relation $\Gamma \models e : \tau$, in
analogy to $\models \varphi$, such that soundness holds.
Such a relation is usually called a \emph{logical relation}. In the rest of
this chapter, we will define a logical relation for the simply-typed lambda
calculus, give a proof of soundness, and show that the logical relation can be
used to establish various interesting properties, such as type safety (again)
and the termination of well-typed programs.

%%%%%%%%%%%%%%%%%%%%%%%%%%%%%%%%%%%%%%%%%%%%%%%%%%%%%%%%%%%%%%%%%%%%%%%%%%%%%%%
\section{Denotation of Types}

As with the formulae of propositional logic, we need to assign some sort of
semantic interpretation to the syntactic types of the lambda calculus. What is
an appropriate semantic domain for the denotation of types? When we try to
think of what a type represents, for example as we are writing a program, we
tend to think of the set of (closed) values that inhabit that type. Therefore,
we define the space of \emph{semantic types} as
\[
  \semcatset{Type} = \mathcal{P}(\syncatset{Value}_{\varnothing}),
\]
where $\syncatset{Value}_{\varnothing}$ is the set of all closed values.

\begin{figure}[t!!]
\begin{alignat*}{2}
  \Denot{\texttt{Unit}} & = \{\texttt{()}\} \\
  \Denot{\tau_1 \to \tau_2} & = \{ v \mid
    \forall v' \in \Denot{\tau_1}\ldotp v\;v' \in \mathcal{E}\Denot{\tau_2} \} \\\\
  \mathcal{E}R & = \{ e \mid \forall e'\ldotp e \longrightarrow^* e'
  \Rightarrow (\exists v \in R \ldotp e'=v) \lor
    (\exists e'' \ldotp e'\longrightarrow e'') \}
\end{alignat*}
\caption{The denotation of types and the expression closure operator.}
\label{fig:logicalRelations_denots}
\end{figure}

We define the denotation $\Denot{\cdot} \colon \syncatset{Type} \to
\semcatset{Type}$ in \autoref{fig:logicalRelations_denots}.
We know that $\texttt{()}$ is the only value of type $\texttt{Unit}$,
so we can take the singleton $\{\texttt{()}\}$ as the semantic
interpretation of this type.
Arrow types are a bit more complicated. They represent functions, and the
essence of a function is that it can be applied to an argument. We will exploit
this intuition in the denotation of arrows by saying that a value behaves like
an arrow if, when applied to a semantically well-typed value, the result is
also semantically well-typed. However, there is one problem: an application is
not a value, so it cannot be in the denotation of the result type.
We somehow need to expand the denotation to all expressions.
For this purpose, we define the expression closure operator
$\mathcal{E} \colon \semcatset{Type} \to
  \mathcal{P}(\syncatset{Expr}_{\varnothing})$.

Expressions which are not values can be reduced.
Thus, given a semantic type $R$, the closure $\mathcal{E} R$
should at the very least contain all the expression that reduce
to a value in $R$. What of expressions that do not terminate? Since
type safety permitted non-termination, we might want to do the same in
$\mathcal{E}$.
Compared to the definition of safety, given as
\[
  \Safe{e} \iff
  \forall e'\ldotp e \longrightarrow^* e'
    \Rightarrow (\exists v \in \syncatset{Value}\ldotp e'=v) \lor
      (\exists e'' \ldotp e'\longrightarrow e''),
\]
$\mathcal{E} R$ is nearly identical, but additionally requires that if the
expression reduces to a value, then this value is in $R$.

As an aside, there is a certain asymmetry in the denotations for
$\texttt{Unit}$ and $\tau_1\to\tau_2$. The former corresponds to the
introduction rule for type $\texttt{Unit}$ and uses the unit value constructor
$\texttt{()}$, while the latter corresponds to the elimination rule for
arrows and uses application.
In our case, we could equivalently define the denotation of an arrow as
\[
  \Denot{\tau_1 \to \tau_2} = \{ \lambda x\ldotp e \mid
  \forall v \in \Denot{\tau_1}\ldotp
    \Subst{e}{v}{x} \in \mathcal{E}\Denot{\tau_2} \}.
\]
This version is similar to the introduction rule for the arrow type, and
requires values of the shape $\lambda x\ldotp e$. The problem with this
formulation is that it is less abstract, and therefore, fails to account for
extensions to the calculus, such as recursive functions. Similarly, we could
try to define a denotation for $\texttt{Unit}$ in the elimination style.
In a sense, this is a degenerate case, as there are no elimination rules for
$\texttt{Unit}$. Therefore, we could place no additional conditions on the
values in the denotation and set
$\Denot{\texttt{Unit}} = \syncatset{Value}_{\varnothing}$. Although
unintuitive, this definition would not compromise any of the theorems
that we will see in this chapter.

%%%%%%%%%%%%%%%%%%%%%%%%%%%%%%%%%%%%%%%%%%%%%%%%%%%%%%%%%%%%%%%%%%%%%%%%%%%%%%%
\section{The Logical Relation}

Equipped with the denotation of types, we would like to define the logical
relation $\Gamma \models e : \tau$. The sole remaining difficulty is that the
expression $e$ might be open, and so we cannot use $\mathcal{E}\Denot{\tau}$
directly, as it only contains closed expressions. We can close $e$ by
substituting closed values for its free variables, and check if the resulting
expression is in $\mathcal{E}\Denot{\tau}$ for any assignment of values.
However, given an expression that is well-typed in $\Gamma$, this substitution
should not be able to produce something that is clearly ill-behaved,
\emph{e.g.}, by substituting $\texttt{()}$ as the left-hand side of function
application. As we expect $\Gamma$ to assign a type to every free variable of
$e$, we can use the denotation of this type as the source of values that are of
the correct semantic type. Therefore, we can interpret the entire environment
as the set of substitutions that are pointwise semantically well-typed,
\emph{i.e.},
\[
  \Denot{\Gamma} = \{\gamma : \dom(\Gamma) \to \syncatset{Value}_{\varnothing}
    \mid \forall x\ldotp \gamma(x) \in \Denot{\Gamma(x)}\}.
\]
Now we can define the logical relation as
\[
  \Gamma \models e \;:\; \tau \iff
   \forall \gamma \in \Denot{\Gamma}\ldotp
   \gamma^*(e) \in \mathcal{E}\Denot{\tau}.
\]

%%%%%%%%%%%%%%%%%%%%%%%%%%%%%%%%%%%%%%%%%%%%%%%%%%%%%%%%%%%%%%%%%%%%%%%%%%%%%%%
\section{Soundness}

To obtain any theorems about syntactically well-typed programs by using the
logical relation, we must first establish soundness. This step is so essential
that this theorem is often called the \emph{Fundamental Property} of the
logical relation.

\begin{theorem}[Fundamental Property]
  If $\Gamma \vdash e \;:\; \tau$ then $\Gamma \models e \;:\; \tau$.
\end{theorem}

The other necessary ingredient is to ask whether our logical relation is
\emph{adequate} for proving the properties that we are interested in. After
all, a relation that holds for all triples of environments, expressions and
types trivially satisfies the Fundamental Property, but is hardly useful. In
order to obtain type safety, we state the adequacy theorem as below.

\begin{theorem}[Adequacy]
  For any $e \in \mathcal{E}R$ we have $\Safe{e}$.
\end{theorem}

Given the similarity between the definitions of $\mathcal{E}R$ and $\rawSafe$,
it is not hard to see that adequacy holds. As a corollary of the Fundamental
Property and adequacy, we get type safety.

\begin{theorem}[Type safety]
  If $\emptyset \vdash e : \tau$ then $\Safe{e}$.
\end{theorem}

Before we turn to the proof of the Fundamental Property, we will state
two preliminary lemmas about the expression closure.

\begin{lemma}\label{lem:lr-val-in-eclo}
  For any semantic type $R$ and value $v \in R$
  we have $v \in \mathcal{E}R$.
\end{lemma}

\begin{lemma}\label{lem:lr-eclo-red}
  If $e \longrightarrow e'$ and $e' \in \mathcal{E}R$ then $e \in \mathcal{E}R$.
\end{lemma}

\autoref{lem:lr-val-in-eclo} lets us show that a value is in $\mathcal{E} R$ by
directly using the semantic type $R$. It follows directly from the definition of
$\mathcal{E}$.
\autoref{lem:lr-eclo-red} says that $\mathcal{E} R$ is
preserved by backward reduction. The seemingly unusual direction of the
reduction is helpful for proving that an expression is in $\mathcal{E} R$,
since by applying this lemma it is enough to show that it is in $\mathcal{E} R$
after performing a reduction step.
We can prove this lemma by unfolding the definition of $\mathcal{E}$ and
utilizing the fact that the reduction relation is deterministic.

In order to prove the Fundamental Property,
we show a series of \emph{compatibility lemmas}.
Each of them corresponds to one typing rule.
We present the proofs of compatibility for the rules for $\lambda$-abstraction
and application.

\begin{lemma}
  $\Gamma \models () \;:\; \mathtt{Unit}$.
\end{lemma}

\begin{lemma}
  If $(x:\tau) \in \Gamma$ then
  $\Gamma \models x \;:\; \tau$.
\end{lemma}

\begin{lemma}
  If $\Gamma, x:\tau_1 \models e \;:\; \tau_2$
  then $\Gamma \models \lambda x.e \;:\; \tau_1 \to \tau_2$.
\end{lemma}
\begin{proof}
  Take any $\gamma \in \Denot{\Gamma}$;
    to show: $\gamma^{*}\lambda x.e
      = \lambda x.\gamma^{*} e
      \in \mathcal{E}\Denot{\tau_1 \to \tau_2}$.\\
  By \autoref{lem:lr-val-in-eclo} it suffices to show 
    $\lambda x.\gamma^{*} e
      \in \Denot{\tau_1 \to \tau_2}$.\\
  Take any $v \in \Denot{\tau_1}$;
    to show: $(\lambda x.\gamma^{*} e)\;v
      \in \mathcal{E}\Denot{\tau_2}$. \\
  By \autoref{lem:lr-eclo-red} it suffices to show
    $(\gamma^{*} e)\{v/x\}
    = \gamma[x\mapsto v]^* e \in \mathcal{E}\Denot{\tau_2}$. \\
  By assumption, it suffices to show
    $\gamma[x\mapsto v] \in \Denot{\Gamma, x : \tau_1}$,
  which is easy to verify by case analysis on variables.
\end{proof}

For convenience, we prove the compatibility lemma for application
for closed terms first.

\begin{lemma}\label{lem:lr-compat-app-cl}
  If $e_1 \in \mathcal{E}\Denot{\tau_2 \to \tau_1}$
  and $e_2 \in \mathcal{E}\Denot{\tau_2}$
  then $e_1\;e_2 \in \mathcal{E}\Denot{\tau_1}$.
\end{lemma}
\begin{proof}
  Take any $e'$ such that $e_1\;e_2 \longrightarrow^* e'$.
  We have the following three cases.
  \begin{enumerate}[label=(\roman*)]
  \item $e_1 \longrightarrow^* e_1'$
    (where $e_1'$ is not a value) and $e' = e_1'\;e_2$.
    By assumption we have $e_1' \longrightarrow e_1''$,
    and therefore, $e' \longrightarrow e_1''\;e_2$.
  \item $e_1 \longrightarrow^* v_1$, $e_2\longrightarrow e_2'$
    (where $e_2'$ is not a value) and $e' = v_1\;e_2'$.
    By assumption we have $e_2' \longrightarrow e_2''$,
    and therefore, $e' \longrightarrow v_1\;e_2''$.
  \item $e_1 \longrightarrow^* v_1$, $e_2 \longrightarrow^* v_2$
    and $v_1\;v_2 \longrightarrow^* e'$.
    By assumption we have $v_1\in\Denot{\tau_2\to\tau_1}$
    and $v_2\in\Denot{\tau_2}$,
    so by the definition of $\Denot{\tau_2\to\tau_1}$
    we have $v_1\;v_2 \in \mathcal{E}\Denot{\tau_1}$.
    Since $v_1\;v_2 \longrightarrow^* e'$ we can conclude the proof.
    \qedhere
  \end{enumerate}
\end{proof}

\begin{lemma}
  If $\Gamma \models e_1 \;:\; \tau_2 \to \tau_1 $
  and $\Gamma \models e_2 \;:\; \tau_2$
  then $\Gamma \models e_1\;e_2 \;:\; \tau_1$.
\end{lemma}
\begin{proof}
  Directly from \autoref{lem:lr-compat-app-cl}.
\end{proof}

Using the compatibility lemmas, the proof of the Fundamental Property is
a straightforward induction on the typing derivation.

%%%%%%%%%%%%%%%%%%%%%%%%%%%%%%%%%%%%%%%%%%%%%%%%%%%%%%%%%%%%%%%%%%%%%%%%%%%%%%%
\section{Termination of the Simply-Typed Lambda Calculus}

Although we used the logical relation to prove type safety, most of our
definitions were based on much more general intuitions about the semantics, and
only the expression closure operator $\mathcal{E}$ at all resembled the desired
property.
This gives us hope that by changing its definition, we can show other
properties with only minimal modifications to the proofs pertaining to the
logical relation.
One such property is \emph{termination}, which we can state as follows.

\begin{theorem}[Termination]
  If $\emptyset \vdash e \;:\; \tau$ then
  there exists $v$ such that $e \longrightarrow^* v$.
\end{theorem}

We can change the definition of the expression closure operator to encode
termination, with the additional condition that the resulting value is of the
appropriate semantic type:
\[
  \mathcal{E}R = \{ e \mid \exists v \in R\ldotp e \longrightarrow^* v \}.
\]
The denotation of types remains the same. However, we need to change our notion
of adequacy to match the theorem to be proven.

\begin{theorem}[Adequacy]
  For any $e \in \mathcal{E}R$ there exists $v$ such that
  $e \longrightarrow^* v$.
\end{theorem}

The statements of the other theorems and lemmas are unchanged. Let us now
examine how their proofs are affected. We still want the two lemmas about
$\mathcal{E}$ to hold. The proofs do have to be redone, but they are no
harder than before.

\begin{lemma}\label{lem:lr-val-in-eclo-sn}
  For any semantic type $R$ and value $v \in R$
  we have $v \in \mathcal{E}R$.
\end{lemma}

\begin{lemma}\label{lem:lr-eclo-red-sn}
  If $e \longrightarrow e'$ and $e' \in \mathcal{E}R$ then $e \in \mathcal{E}R$.
\end{lemma}

By induction, we can also generalize \autoref{lem:lr-eclo-red-sn} to reduction
in many steps.

\begin{lemma}\label{lem:lr-eclo-red-rtc-sn}
  If $e \longrightarrow^* e'$ and $e' \in \mathcal{E}R$ then $e \in \mathcal{E}R$.
\end{lemma}

In the first three compatibility lemmas---for the unit value, variables
and $\lambda$-abstractions---we can use \autoref{lem:lr-val-in-eclo-sn} and
never unfold the definition of $\mathcal{E}$, so the original proofs still go
through.
As the (closed version of) compatibility for application uses the
expression closure directly, we have to revisit its proof.

\begin{lemma}\label{lem:lr-compat-app-cl-sn}
  If $e_1 \in \mathcal{E}\Denot{\tau_2 \to \tau_1}$
  and $e_2 \in \mathcal{E}\Denot{\tau_2}$
  then $e_1\;e_2 \in \mathcal{E}\Denot{\tau_1}$.
\end{lemma}
\begin{proof}
  From the assumptions we have $v_1 \in \Denot{\tau_2 \to \tau_1}$ such
  that $e_1 \longrightarrow^* v_1$, and $v_2 \in \Denot{\tau_2}$ such
  that $e_2 \longrightarrow^* v_2$. \\
  By \autoref{lem:lr-eclo-red-rtc-sn} and properties of
  reduction it suffices to show $v_1\;v_2 \in \mathcal{E}\Denot{\tau_1}$,
  which follows from the definition of $\Denot{\tau_2 \to \tau_1}$
  and the assumptions.
\end{proof}

To prove termination, we did not have to start from scratch: we only modified
a few of the proofs to account for the new definition of $\mathcal{E}$.
This already shows us some of the versatility of logical relations as
compared to the traditional techniques we have used in the previous chapter.
However, the two different proofs of the compatibility lemma for application
were quite different from each other, and somewhat convoluted due to requiring
tedious reasoning about multi-step reductions on subexpressions.
In the next chapter we will see how to fix these flaws, and also make the
logical relation genuinely modular with respect to the considered property.

%%%%%%%%%%%%%%%%%%%%%%%%%%%%%%%%%%%%%%%%%%%%%%%%%%%%%%%%%%%%%%%%%%%%%%%%%%%%%%%
\section{Further Reading}
