\chapter{Logical Relations}

%%%%%%%%%%%%%%%%%%%%%%%%%%%%%%%%%%%%%%%%%%%%%%%%%%%%%%%%%%%%%%%%%%%%%%%%%%%%%%%
\section{Syntactic Provability and Semantic Truth}

% Some intuitions from propositional logic go here

\newcommand\Denot[1]{\llbracket#1\rrbracket}
\begin{alignat*}{2}
  \varphi & \Coloneqq \top \mid \bot \mid \varphi \to \varphi
\end{alignat*}

$\Denot{\cdot} \colon \syncatset{Formula} \to \mathcal{D}$
\begin{alignat*}{2}
  \Denot{\top} & = \texttt{true} \\
  \Denot{\bot} & = \texttt{false} \\
  \Denot{\varphi \to \psi} & = \Denot{\varphi} \Rightarrow \Denot{\psi}
\end{alignat*}

%%%%%%%%%%%%%%%%%%%%%%%%%%%%%%%%%%%%%%%%%%%%%%%%%%%%%%%%%%%%%%%%%%%%%%%%%%%%%%%
\section{Denotation of Types}

$\semcatset{Type} = \mathcal{P}(\syncatset{Value}_{\varnothing})$
(sets of closed values)

$\Denot{\cdot} \colon \syncatset{Type} \to \semcatset{Type}$ ---
  denotation of a type --- set of values of that type
\begin{alignat*}{2}
  \Denot{\texttt{Unit}} & = \{\texttt{()}\} \\
  \Denot{\tau_1 \to \tau_2} & = \{ v \mid
    \forall v' \in \Denot{\tau_1}\ldotp v\;v' \in \mathcal{E}\Denot{\tau_2} \}
\end{alignat*}

$\mathcal{E} \colon \semcatset{Type} \to
  \mathcal{P}(\syncatset{Expr}_{\varnothing})$

$\mathcal{E}R = \{ e \mid \forall e'\ldotp e \longrightarrow^* e'
  \Rightarrow (\exists v \in R \ldotp e'=v) \lor
    (\exists e'' \ldotp e'\longrightarrow^*e'') \}$

%%%%%%%%%%%%%%%%%%%%%%%%%%%%%%%%%%%%%%%%%%%%%%%%%%%%%%%%%%%%%%%%%%%%%%%%%%%%%%%
\section{The Logical Relation}

% From denotation of types to open expressions

$\Denot{\Gamma} = \{\gamma \mid \forall x\ldotp \gamma(x) \in \Denot{\Gamma(x)}\}$
($\gamma \in \Denot{\Gamma} \Longleftrightarrow \forall x\ldotp \gamma(x) \in \Denot{\Gamma(x)} $)

$\gamma \colon X \to \syncatset{Value}_{\varnothing}$

$\Gamma \models e \;:\; \tau \Longleftrightarrow
  \forall \gamma \in \Denot{\Gamma}\ldotp
  \gamma^*(e) \in \mathcal{E}\Denot{\tau}$

%%%%%%%%%%%%%%%%%%%%%%%%%%%%%%%%%%%%%%%%%%%%%%%%%%%%%%%%%%%%%%%%%%%%%%%%%%%%%%%
\section{Soundness}

\begin{theorem}[Fundamental Property]
  If $\Gamma \vdash e \;:\; \tau$ then $\Gamma \models e \;:\; \tau$.
\end{theorem}

\begin{theorem}[Adequacy]
  For any $e \in \mathcal{E}R$ we have $\Safe{e}$.
\end{theorem}

\begin{theorem}[Type safety]
  If $\emptyset \vdash e : \tau$ then $\Safe{e}$.
\end{theorem}

\begin{lemma}\label{lem:lr-val-in-eclo}
  For any semantic type $R$ and value $v \in R$
  we have $v \in \mathcal{E}R$.
\end{lemma}

\begin{lemma}\label{lem:lr-eclo-red}
  If $e \longrightarrow e'$ and $e' \in \mathcal{E}R$ then $e \in \mathcal{E}R$.
\end{lemma}

In order to prove Fundamental Property
we show series of \emph{compatibility lemmas}.
Each of them corresponds to one typing rule.

\begin{lemma}
  $\Gamma \models () \;:\; \mathtt{Unit}$
\end{lemma}

\begin{lemma}
  If $(x:\tau) \in \Gamma$ then
  $\Gamma \models x \;:\; \tau$.
\end{lemma}

\begin{lemma}
  If $\Gamma, x:\tau_1 \models e \;:\; \tau_2$
  then $\Gamma \models \lambda x.e \;:\; \tau_1 \to \tau_2$.
\end{lemma}
\begin{proof}
  Take any $\gamma \in \Denot{\Gamma}$;
    to show: $\gamma^{*}\lambda x.e
      = \lambda x.\gamma^{\Uparrow_x*} e
      \in \mathcal{E}\Denot{\tau_1 \to \tau_2}$.\\
  By \autoref{lem:lr-val-in-eclo} it suffices to show 
    $\lambda x.\gamma^{\Uparrow_x*} e
      \in \Denot{\tau_1 \to \tau_2}$.\\
  Take any $v \in \Denot{\tau_1}$;
    to show: $(\lambda x.\gamma^{\Uparrow_x*} e)\;v
      \in \mathcal{E}\Denot{\tau_2}$. \\
  By \autoref{lem:lr-eclo-red} it suffices to show
    $(\gamma^{\Uparrow_x*} e)\{v/x\}
    = (v \triangleleft_x \gamma)^* e \in \mathcal{E}\Denot{\tau_2}$. \\
  By assumption, it suffices to show
    $v \triangleleft_x \gamma \in \Denot{\Gamma, x : \tau_1}$,
  which is easy to verify by case analysis on variable.
\end{proof}

For convenience, we prove compatibility lemma for application
for closed terms first.

\begin{lemma}\label{lem:lr-compat-app-cl}
  If $e_1 \in \mathcal{E}\Denot{\tau_2 \to \tau_1}$
  and $e_2 \in \mathcal{E}\Denot{\tau_2}$
  then $e_1\;e_2 \in \mathcal{E}\Denot{\tau_1}$.
\end{lemma}
\begin{proof}
  Pick any $e'$ such that $e_1\;e_2 \longrightarrow^* e'$.
  We have three, the following cases.
  \begin{enumerate}[label=(\roman*)]
  \item $e_1 \longrightarrow^* e_1'$ (not a value) and $e' = e_1'\;e_2$.
    By assumption we have $e_1' \longrightarrow e_1''$
    and therefore, $e' \longrightarrow e_1''\;e_2$.
  \item $e_1 \longrightarrow^* v_1$, $e_2\longrightarrow e_2'$
    (not a value) and $e' = v_1\;e_2'$.
    By assumption we have $e_2' \longrightarrow e_2''$
    and therefore, $e' \longrightarrow v_1\;e_2''$.
  \item $e_1 \longrightarrow^* v_1$, $e_2 \longrightarrow^* v_2$
    and $v_1\;v_2 \longrightarrow^* e'$.
    By assumption we have $v_1\in\Denot{\tau_2\to\tau_1}$
    and $v_2\in\Denot{\tau_2}$,
    so by the definition of $\Denot{\tau_2\to\tau_1}$
    we have $v_1\;v_2 \in \mathcal{E}\Denot{\tau_1}$.
    Since $v_1\;v_2 \longrightarrow^* e'$ we can conclude the proof.
    \qedhere
  \end{enumerate}
\end{proof}

\begin{lemma}
  If $\Gamma \models e_1 \;:\; \tau_2 \to \tau_1 $
  and $\Gamma \models e_2 \;:\; \tau_2$
  then $\Gamma \models e_1\;e_2 \;:\; \tau_1$.
\end{lemma}
\begin{proof}
  Directly from \autoref{lem:lr-compat-app-cl}.
\end{proof}

\medskip

Using Compatibility Lemmas proof of Fundamental Property is straightforward induction.

%%%%%%%%%%%%%%%%%%%%%%%%%%%%%%%%%%%%%%%%%%%%%%%%%%%%%%%%%%%%%%%%%%%%%%%%%%%%%%%
\section{Termination of Simply-Typed Lambda Calculus}

\begin{theorem}{Termination}
  If $\emptyset \vdash e \;:\; \tau$ then
  there exists $v$ such that $e \longrightarrow^* v$.
\end{theorem}

To prove termination we will change definition of $\mathcal{E}$ to reflect that fact.

$\mathcal{E}R = \{ e \mid \exists v \in R\ldotp e \longrightarrow^* v \}$

%%%%%%%%%%%%%%%%%%%%%%%%%%%%%%%%%%%%%%%%%%%%%%%%%%%%%%%%%%%%%%%%%%%%%%%%%%%%%%%
\section{Further Reading}
