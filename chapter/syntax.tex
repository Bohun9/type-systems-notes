\chapter{Syntax}

%%%%%%%%%%%%%%%%%%%%%%%%%%%%%%%%%%%%%%%%%%%%%%%%%%%%%%%%%%%%%%%%%%%%%%%%%%%%%%%
\section{Grammars and Abstract Syntax Trees}

%%%%%%%%%%%%%%%%%%%%%%%%%%%%%%%%%%%%%%%%%%%%%%%%%%%%%%%%%%%%%%%%%%%%%%%%%%%%%%%
\section{The Classic Approach to Variable Binding}

%%%%%%%%%%%%%%%%%%%%%%%%%%%%%%%%%%%%%%%%%%%%%%%%%%%%%%%%%%%%%%%%%%%%%%%%%%%%%%%
\section{Variable Binding via Indexed Families of Sets}

% Indexed family of sets of terms
\newcommand\ITerm[1]{\mathrm{Term}_{#1}}
% Variable constructor
\newcommand\ITVar[1]{\ulcorner #1 \urcorner}
% Successor (constructor of X+1)$
\newcommand\ITSucc{\mathsf{s}}
% Extending set by one element
\newcommand\ITInc[1]{#1\!+\!1}

$\ITerm{X}$ is smallest set, such that:

\begin{itemize}
\item $\forall x \in X\ldotp \ITVar{x} \in \ITerm{X}$
\item $\forall t \in \ITerm{X+1}\ldotp \lambda t \in \ITerm{X}$
\item $\forall t_1, t_2 \in \ITerm{X}\ldotp t_1\;t_2 \in \ITerm{X}$
\end{itemize}

$\ITInc{X}$ is isomorphic to \texttt{Maybe} or \texttt{option}
and is defined as:

\begin{itemize}
\item $0 \in \ITInc{X}$
\item $\forall x \in X\ldotp \ITSucc\;x \in \ITInc{X}$
\end{itemize}

$^* \colon (X \to \ITerm{Y}) \to \ITerm{X} \to \ITerm{Y}$ (also known as \texttt{bind})

\begin{eqnarray*}
  f^*\ITVar{x}   & = & f\;x \\
  f^*(\lambda t) & = & \lambda (f^{\Uparrow*} \; t) \\
  f^*(t_1\;t_2)  & = & (f^* \; t_1) \; (f^* \; t_2)
\end{eqnarray*}

$^\Uparrow \colon (X \to \ITerm{Y}) \to \ITInc{X} \to \ITerm{\ITInc{Y}}$

\begin{eqnarray*}
  f^\Uparrow 0 & = & \ITVar{0} \\
  f^\Uparrow (\ITSucc \; x) & = & \ITSucc^\dagger\;(f\;x) \\
\end{eqnarray*}

$^\dagger \colon (X \to Y) \to \ITerm{X} \to \ITerm{Y}$ (also known as \texttt{fmap})

\begin{eqnarray*}
  f^\dagger\ITVar{x}   & = & \ITVar{f\;x} \\
  f^\dagger(\lambda t) & = & \lambda(f^{\uparrow\dagger}\;t) \\
  f^\dagger(t_1\;t_2)  & = & (f^\dagger\;t_1)\;(f^\dagger\;t_2)
\end{eqnarray*}

$\uparrow \colon (X \to Y) \to \ITInc{X} \to \ITInc{Y}$
  (also known as \texttt{fmap} but on \texttt{Maybe})

\begin{eqnarray*}
  f^\uparrow 0            & = & 0 \\
  f^\uparrow (\ITSucc\;x) & = & \ITSucc(f^\uparrow \; x)
\end{eqnarray*}

%%%%%%%%%%%%%%%%%%%%%%%%%%%%%%%%%%%%%%%%%%%%%%%%%%%%%%%%%%%%%%%%%%%%%%%%%%%%%%%
\section{Further Reading}
