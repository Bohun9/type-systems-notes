\chapter{Syntax}

%%%%%%%%%%%%%%%%%%%%%%%%%%%%%%%%%%%%%%%%%%%%%%%%%%%%%%%%%%%%%%%%%%%%%%%%%%%%%%%
\section{Grammars and Abstract Syntax Trees}

Throughout the remainder of these lecture notes, we will specify the abstract
syntax of various elements of the considered calculi in the form of a grammar,
such as the following grammar of the plain $\lambda$-calculus.
\begin{alignat*}{2}
  \syncatset{Var}  &\ni x, y, z, \dots \span\span \tag{variables} \\
  \syncatset{Term} &\ni t &&\Coloneqq x \mid \lambda x\ldotp t \mid t\;t \tag{terms}
\end{alignat*}
There are only two syntactic categories in this grammar: variables and terms.
For each syntactic category we define a set of all its elements:
$\syncatset{Var}$ is a set of all variables
and $\syncatset{Term}$ is a set of all terms.
Moreover, we assign one or more \emph{metavariables}
to each syntactic category: $x, y, z, \dots$ for variables
and $t$ for terms.
By convention, whenever we write some math variable with a name based on $t$
(\emph{e.g.}, $t'$, $t_1$, $t_2''$) we implicitly assume
that it describes an element of set $\syncatset{Term}$.

Terms are defined by a set of productions given on the
right-hand-side of~$\Coloneqq$.
A term can be a variable,
a $\lambda$-abstraction, or an application of one term to another.
As this grammar is supposed to represent the \emph{abstract syntax trees},
rather than the strings of any particular concrete syntax, we do not consider
parentheses, priorities or associativity. When writing such
terms down, we will use various common conventions, \emph{e.g.}, the
body of a $\lambda$-abstraction extends as far right as possible, and
application is left-associative. This allows us to write
$\lambda x\ldotp t_1\;t_2$ for $\lambda x\ldotp (t_1\;t_2)$
and $t_1\;t_2\;t_3$ for $(t_1\;t_2)\;t_3$.

Note that the presented grammar says nothing about how the variables are
represented. It is convenient to abstract away from such
inconsequential details on paper, and make various assumptions about which
variables can be safely treated as distinct. However, to ground such
intuitions and to give us a good foundation for formal reasoning with
the use of a proof assistant, the following two sections describe
two concrete approaches to variables and variable binding.

%%%%%%%%%%%%%%%%%%%%%%%%%%%%%%%%%%%%%%%%%%%%%%%%%%%%%%%%%%%%%%%%%%%%%%%%%%%%%%%
\section{The Classic Approach to Variable Binding}
\label{sec:syntax_classic}

\newcommand\Preterm{\syncatset{Preterm}}
\newcommand\Term{\syncatset{Term}}
\newcommand\Var{\syncatset{Var}}
\newcommand\FVars[1]{\mathrm{fv}(#1)}
\newcommand\Subst[3]{#1\{#2/#3\}}
\newcommand\Aeq{\equiv_\alpha}

Let $\Var$ be some infinite set of variable names.
Then we can define $\Preterm$ as the smallest set such that
\begin{itemize}
\item $\forall x \in \Var\ldotp x \in \Preterm$,
\item $\forall x \in \Var\ldotp\forall t \in \Preterm\ldotp \lambda x\ldotp t \in \Preterm$,
\item $\forall t_1, t_2 \in \Preterm\ldotp t_1\;t_2 \in \Preterm$.
\end{itemize}
At first glance, this definition appears similar to the grammar of the
$\lambda$-calculus.
However, if $x \neq y$, then $\lambda x\ldotp x$ and $\lambda y\ldotp y$ are
not equal, even though they differ only in the choice of bound variable names,
which does not affect the represented computation. We will define an
equivalence relation, traditionally known as $\alpha$-equivalence, to identify
such terms.

\begin{figure}[t!!]
\[
  \Subst{\cdot}{\cdot}{\cdot} \colon
  \Preterm \times \Var \times \Preterm \partialmap \Preterm
\]
\begin{eqnarray*}
  \Subst{x}{t}{x}          & = & t \\
  \Subst{y}{t}{x}          & = & y, \text{ if } x \ne y \\
  \Subst{(t_1\;t_2)}{t}{x} & = & (\Subst{t_1}{t}{x})\;(\Subst{t_1}{t}{x}) \\
  \Subst{(\lambda x\ldotp t')}{t}{x} & = & \lambda x\ldotp t' \\
  \Subst{(\lambda y\ldotp t')}{t}{x} & = & \lambda y\ldotp \Subst{t'}{t}{x}, \text{ if } x \ne y \land y \not\in \FVars{t}
\end{eqnarray*}
\caption{The substitution operation for preterms.}
\label{fig:syntax_preterm_subst}
\end{figure}

We are going to need a way to rename the variables occurring in preterms.
More generally, we can define the substitution of entire preterms for
variables as a partial function, as shown in
\autoref{fig:syntax_preterm_subst}.
The case for $\lambda$-abstraction is tricky, since we need to be careful
to avoid \emph{variable capture}.
For example, in the substitution $\Subst{(\lambda y\ldotp x)}{y}{x}$,
we cannot simply substitute under the binder. If we did, we would get
$\lambda y \ldotp y$, which is clearly incorrect---the free variable $y$ was
captured by the abstraction!
Thus, we define $\Subst{(\lambda y\ldotp t')}{t}{x}$ only when $y$ is not in
the set of free variables of $t$, denoted as $\FVars{t}$. Despite being
partial, this notion of substitution is sufficient for the purpose of defining
$\alpha$-equivalence.

\begin{figure}[t!!]
\begin{mathpar}
  \inferrule{ }
            {t \Aeq t}

  \inferrule{t_2 \Aeq t_1}
            {t_1 \Aeq t_2}

  \inferrule{t_1 \Aeq t_2 \\ t_2 \Aeq t_3}
            {t_1 \Aeq t_3}

  \inferrule{ }
            {x \Aeq x}

  \inferrule{t_1 \Aeq t_1' \\ t_2 \Aeq t_2'}
            {t_1\;t_2 \Aeq t_1'\;t_2'}

  \inferrule{t_1 \Aeq t_2}
            {\lambda x.t_1 \Aeq \lambda x.t_2}

  \inferrule{y \not\in \FVars{t}}
            {\lambda x.t \Aeq \lambda y.\Subst{t}{y}{x}}

\end{mathpar}
\caption{The $\alpha$-equivalence relation.}
\label{fig:syntax_alpha_equiv}
\end{figure}

The $\alpha$-equivalence relation is defined in
\autoref{fig:syntax_alpha_equiv}.
It has the three usual rules that make it an equivalence relation:
reflexivity, symmetry and transitivity. Additionally, the relation
is a congruence: it is compatible with the preterm constructors
thanks to the three structural rules.\footnote{
  The rule for variables is somewhat redundant, as it is just a special case
  of reflexivity.
} Finally, the last presented rule allows the variable bound by a
$\lambda$-abstraction to be renamed, as long as it does not result in the
capture of any of the free variable in its body.

We can now define the set of terms as the quotient set
\[ \Term = \Preterm / {\Aeq}, \]
which is the set of all equivalence classes of $\Aeq$. It is customary to
simply write $t$ in place of $[t]_{\Aeq}$, and only consider terms up to
$\alpha$-equivalence. Substitution for terms can be defined as a total
function, as capture can be avoided by the use of variable renaming.
We can lift the substitution operation for preterms to terms by using it on the
representatives of the $\alpha$-equivalence classes. Proving that this
definition is well-behaved is left as an exercise.

%%%%%%%%%%%%%%%%%%%%%%%%%%%%%%%%%%%%%%%%%%%%%%%%%%%%%%%%%%%%%%%%%%%%%%%%%%%%%%%
\section{Variable Binding via Indexed Families of Sets}

% Indexed family of sets of terms
\newcommand\ITerm[1]{\syncatset{Term}_{#1}}
% Variable constructor
\newcommand\ITVar[1]{\ulcorner\! #1 \!\urcorner}
% Successor (constructor of X+1)$
\newcommand\ITSucc{\mathsf{s}}
% Extending set by one element
\newcommand\ITInc[1]{#1\!+\!1}

While the treatment of variable binding in the previous section is close
to intuitive meaning, the construction is bit involved and not very convenient
when we work with proof assistants, like Coq.
In this section we present another approach that do not require
any quotient sets.
In order to avoid quotient sets and $\alpha$-equivalence our representation
should be \emph{nameless}, \emph{i.e.}, bound variables should not have
any names.

One well-known example of a nameless approach is a representation based on
de Bruijn indices~\citep{deBruijn72}.
The idea is to uniquely represent a bound variable by a number (index),
that describes how many other binders we should skip to find
a corresponding binder.
For instance, term $\lambda x.\lambda y.x\;y$ will be
represented as $\lambda\lambda 1\;0$.
Two things should be noted here.
First, the lambda abstraction does not store any name,
because the representation is nameless and
bound variables are matched with their binders not by name,
but by counting.
Secondly, it is not clear how to represent free variables.
Basically, there are two approaches.
\begin{itemize}
\item We can treat indices as free, when they are too big to be bound.
  As this seems to be natural, it has some drawbacks.
  Usually, we identify free variables by their names.
  These names have no additional structure,
  but have some meaning to the programmer.
  In this plain de Bruijn representation free variables
  have no names and we impose some order on then (order on natural numbers)
  that have no counterpart on our intuitive understanding of variables.
\item We can have separate notion of free variables (represented by names)
  and bound variables (represented by indices).
  Such an approach is called
  \emph{locally nameless}~\citep{AydemirCPPW08,Chargueraud12}
  and was successfully applied to some larger Coq or Agda formalizations.
  However, in this representation we still have to deal with indices
  too big to be bound, and in order to do so,
  we work with \emph{locally closed terms}, \emph{i.e.},
  terms that do not have such a too big indices.
  Such an additional ``locally closed'' relation introduces some
  unnecessary clutter into a formalization.
\end{itemize}

In this section we present another nameless approach,
that combines advantages of both mentioned approaches.
We start with the observation that in most cases, when we work
with open terms, there is a some kind of scope or an environment
that provides a meaning of free variables.
In such cases we allow only terms where all free variables
are in this scope.
For instance, evaluation function $\texttt{eval}\;\rho\;e$
accepts only expressions where for each free variable~$x$,
its meaning $\rho(x)$ is defined in the environment,
or typing relation $\Gamma\vdash e\;:\:\tau$
(see \autoref{ch:simple-types})
requires that all free variables have assigned type in~$\Gamma$.
Another extreme example is when we work with closed terms:
their scope is empty and no free variables are allowed.

With this intuition in mind, we can construct a set~$\ITerm{X}$,
for each set~$X$, that will be called a~set of
\emph{potentially free} variables.
We can think about this set of potentially free variables
as a scope of the term.
For example $\ITerm{\varnothing}$ contains all closed terms
(without any free variables),
while $\ITerm{\Var}$ is a set (isomorphic to a set)
of terms from the previous section.

We will define $\ITerm{X}$ inductively.
Before going into details, let us see how we can represent a variable binding.
Obviously, we can find some lambda-abstractions in $\ITerm{X}$.
Because we are aiming for a nameless representation,
these lambda-abstraction have the form $\lambda t$,
where the variable bound by a lambda has no name.
What is a set of potentially free variables of the body~$t$?
It is a set~$X$ extended with this one extra variable bound by the lambda.
We will write $\ITInc{X}$ for such a set,
and $0$ for this new element.
But $X$ may be an arbitrary set, in particular, it can already contain element~$0$.
Therefore, in the set $\ITInc{X}$
we wrap all elements of $X$ around the constructor $\ITSucc$
in order to distinguish them from $0$.
Formally, $\ITInc{X}$ has the following inductive definition.

\begin{defin}
  For any set $X$ we define $\ITInc{X}$ as the smallest set, such that:
  \begin{thmenumerate}
  \item $0 \in \ITInc{X}$,
  \item for each $x\in X$ we have $\ITSucc\;x \in X$.
  \end{thmenumerate}
\end{defin}

Note that $\ITInc{X}$ is isomorphic to \texttt{Maybe} or \texttt{option}
known from functional programming.
Now, we can formally define set of terms~$\ITerm{X}$.

\begin{defin}
  For any set $X$ we define $\ITerm{X}$ as the smallest set, such that:
  \begin{thmenumerate}
  \item for each $x \in X$ a term $\ITVar{x}$ that represents a variable
    is a valid term ($\ITVar{x} \in \ITerm{X}$);
  \item for each $t \in \ITerm{\ITInc{X}}$
    a lambda-abstraction $\lambda t$ is a valid term ($\lambda t \in \ITerm{X}$);
  \item for each $t_1,t_2 \in \ITerm{X}$ an application $t_1\;t_2$
    is a valid term ($t_1\;t_2 \in \ITerm{X}$).
  \end{thmenumerate}
\end{defin}

Since a set of potentially free variables may be an arbitrary set,
in particular, it can be also a set of terms (we may have $\ITerm{\ITerm{X}}$).
In order to avoid some notational ambiguity, we explicitly wrap variables around
a constructor $\ITVar{x}$.

An interesting side note is a connection of presented term representation with
de Bruijn indices.
If we treat $\ITSucc\;n$ as a successor of $n$, and write $\ITSucc\;0$ as $1$,
$\ITSucc(\ITSucc\;0)$ as $2$, \emph{etc.},
then for closed terms we obtain exactly de Bruijn indices!
For example term $\lambda x.\lambda y. x\;y$
would be written as $\lambda\lambda \ITVar{1}\;\ITVar{0}$.

The next ingredient of presented approach to syntax is
a definition of a substitution.
Usually we substitute a term for a one distinguished variable.
We can see set $\ITInc{X}$ as a set with one distinguished value $0$,
so we define a substitution as a function of the following signature.
\[
	\cdot\{\cdot\}\colon
  \ITerm{\ITInc{X}} \times \ITerm{X} \to \ITerm{X}
\]
Note that if we substitute term $t'$ in $t$ (written $t\{t'\}$)
then all free variables in $t$ are modified:
we substitute $t'$ for variable $0$,
and we change each $\ITSucc\;x$ into $x$.
Instead of defining substitution directly,
we start with defining more general \emph{simultaneous substitution}
for all potentially free variables at once.

\begin{figure}[t]
  \begin{center}
    \begin{tabular}{p{0.44\textwidth}p{0.49\textwidth}}
      {$\cdot^\uparrow \colon (X \to Y) \to \ITInc{X} \to \ITInc{Y}$
      \begin{alignat*}{2}
        f^\uparrow 0            & = 0 \\
        f^\uparrow (\ITSucc\;x) & = \ITSucc\;(f \; x)
      \end{alignat*}} &
      {$\cdot^\Uparrow \colon (X \to \ITerm{Y}) \to \ITInc{X} \to \ITerm{\ITInc{Y}}$
      \begin{alignat*}{2}
        g^\Uparrow 0              & = \ITVar{0} \\
        g^\Uparrow (\ITSucc \; x) & = \ITSucc^\dagger(g\;x)
      \end{alignat*}}
    \end{tabular}

    \begin{tabular}{p{0.44\textwidth}p{0.49\textwidth}}
      {$\cdot^\dagger \colon (X \to Y) \to \ITerm{X} \to \ITerm{Y}$
      \begin{alignat*}{2}
        f^\dagger\ITVar{x}   & = \ITVar{f\;x} \\
        f^\dagger(\lambda t) & = \lambda(f^{\uparrow\dagger}\;t) \\
        f^\dagger(t_1\;t_2)  & = (f^\dagger t_1)\;(f^\dagger t_2)
      \end{alignat*}} &
      {$\cdot^* \colon (X \to \ITerm{Y}) \to \ITerm{X} \to \ITerm{Y}$
      \begin{alignat*}{2}
        g^*\ITVar{x}   & = g\;x \\
        g^*(\lambda t) & = \lambda (g^{\Uparrow*} \; t) \\
        g^*(t_1\;t_2)  & = (g^* t_1) \; (g^* t_2)
      \end{alignat*}}
    \end{tabular}
    \caption{Renaming and simultaneous substitution together with
      their corresponding lifting functions.}
    \label{fig:syntax_simultaneous_subst}
  \end{center}
\end{figure}

Precise definition of simultaneous substitution (function $\cdot^*$) is
presented in \autoref{fig:syntax_simultaneous_subst}.
In order to substitute for all free variables in term $t\in\ITerm{X}$
we need to describe for each variable in $X$
which term should be substituted for it.
We do so, by providing a \emph{substitution} function $g\colon X\to\ITerm{Y}$
(set $Y$ can be any set, possibly unrelated to $X$).
For each such a function we define simultaneous substitution
$g^*\colon\ITerm{X}\to\ITerm{Y}$.
Substitution $g$ in a variable term $\ITVar{x}$ is just $g(x)$,
as we substitute for all free variables.
For other language constructs we proceed structurally.
The case for $\lambda$-abstraction is interesting,
because we proceed with substitution under the binder.
Therefore, we have to \emph{lift} substitution~$g$ to
a function $g^\Uparrow\colon\ITInc{X}\to\ITerm{\ITInc{Y}}$.
This lifting operation should not affect newly bound variable $0$,
so it maps $0$ to term $\ITVar{0}$.
For other variables (of the form $\ITSucc\;x$)
it returns original $g(x)$
with all variables \emph{shifted} by one (operation $\ITSucc^\dagger$),
as the variable $0$ is going to be bound by the nearest $\lambda$-abstraction.

We define the shifting operation $\ITSucc^\dagger$
as a special case of more general \emph{renaming}.
Given function $f\colon X\to Y$ that renames variables in set $X$
to those in set~$Y$,
we define a function $f^\dagger\colon\ITerm{X}\to\ITerm{Y}$
that applies function $f$ to each free variable in given term.
Similarly to simultaneous substitution, such a simultaneous renaming
proceed structurally for each language construct and
in case of $\lambda$-abstraction
it has to lift renaming~$f$
to a function $f^\uparrow\colon\ITInc{X}\to\ITInc{Y}$.
In case of lifting of renamings
we don't need any auxiliary function,
because renamings map variables to variables, not to terms.

Given substitution function $g\colon X\to\ITerm{Y}$
we can extend it by a term $t\in\ITerm{Y}$ and obtain a substitution
$g[\mapsto t]\colon \ITInc{X}\to\ITerm{Y}$.
Such an extension is defined as follows.
\begin{align*}
  g[\mapsto t]\;0 & = t &
  g[\mapsto t]\;(\ITSucc\;x) & = g\;x
\end{align*}
Now, we are in position to define a substitution for single variable
\[ t\{t'\} = \eta[\mapsto t']^*\;t \textrm{,} \]
where $\eta$ is an identity substitution:
\[ \eta\;x = \ITVar{x}\textrm{.} \]
Following our intuition, substitution $t\{t'\}$
substitutes $t'$ for variable $0$
and other variables of the form $\ITSucc\;x$
changes to $\ITVar{x}$.

The reader might notice that simultaneous renaming
and simultaneous substitution
have the same signatures as \texttt{fmap} and \texttt{bind}
functions, known from functional programming (or category theory).
This similarity is not accidental,
as syntax together with simultaneous renaming form a functor
which together with simultaneous substitution form a monad.

\begin{lemma}
  Simultaneous renaming satisfy the following functorial laws
  (where $\id$ is an identity function,
  and $\circ$ denotes a function composition):
  \begin{thmenumerate}
  \item $\id^\dagger = \id$,
  \item $(f_1\circ f_2)^\dagger = f_1^\dagger \circ f_2^\dagger$.
  \end{thmenumerate}
\end{lemma}

\begin{lemma}
  Simultaneous substitution together with unit substitution $\eta$
  satisfy the following monadic laws:
  \begin{thmenumerate}
  \item $g^* \circ \eta = g$,
  \item $\eta^* = \id$,
  \item $g_1^* \circ g_2^* = (g_1^* \circ g_2)^*$.
  \end{thmenumerate}
\end{lemma}

%%%%%%%%%%%%%%%%%%%%%%%%%%%%%%%%%%%%%%%%%%%%%%%%%%%%%%%%%%%%%%%%%%%%%%%%%%%%%%%
\section{Informal Presentation vs Formalization}

Syntax with variable binding is ubiquitous in the theory of programming
languages.
In this chapter we provided a very precise treatment of such syntax
in order to give solid foundations for the rest of the notes,
as well as to outline how such a theory can be formalized
using proof assistants.
In the remainder of these notes we stay
on the informal level regarding variable binding.
However, we believe that formalizing mathematics in proof assistants
is important, especially in the theory of programming languages.
Therefore, we try to present a technical material of these notes
in a way that is not too far from possible formalization,
even if formalizing mathematics is outside of the scope of these notes.

There are two main differences between the informal presentation
and the proposed approach to formalization.
Firstly, the formalization uses nameless representation of variable
binding, while in the informal mathematics, as well as in computer programs
we prefer to use names to connect a variable with its binder.
We see this as a rather minor issue,
that is a reminiscent of the difference between concrete
and abstract syntax of programming languages.
The reader interested in formalizing theory should employ additional
``parsing phase'' to presented definitions and theorems.

The second difference concerns the properties that are built into
term representation.
The nameless representation of this chapter is well-scoped by design:
the set of of terms is parametrized by a set of potentially free variables,
or in other words: a \emph{scope}.
On the other hand,
traditional informal presentation permits terms with arbitrary
free variables,
while being closed or well-scoped is proven \emph{post-hoc}
or externally assumed.
In these notes we stick to traditional approach,
but having the well-scoped formalization in mind.
Unless explicitly stated otherwise,
terms in presented definitions and theorems have some implicit scope
that can be inferred from the context.
The properties that free variables of such terms are in this implicit scope
can be proven, but we usually don't even state such properties
as not to bore the reader.

In order to further narrow the gap between informal presentation
and formalization,
we borrow some notations from indexed families of sets and use them
in the classical approach.
We summarize our unified notations as follows.
\begin{itemize}
\item We write $\Term_\varnothing$ for set of closed terms
  (in a sense of \autoref{sec:syntax_classic}).
  We use the same convention for other syntactic categories,
  \emph{e.g.}, $\syncatset{Expr}_\varnothing$, $\syncatset{Type}_\varnothing$.
\item Given partial function $g\colon\Var \partialmap \Term$
  we define a simultaneous capture-avoiding
  substitution $g^*\colon \Term \to \Term$,
  that for each free variable~$x$ substitutes $g(x)$
  or leaves $x$ intact if $g(x)$ is undefined.
  Note that if we assume Barendregt's variable convention,
  we don't have to define a traditional counterpart of
  lifting operation ($g^\Uparrow$), as it is almost identity operation.
  When we proceed with substitution~$g$ under the binder of $x$,
  then $g^*$ will map $x$ to itself.
\item For possibly partial function~$f$ that maps variables to some
  elements, we write $f[x\mapsto v]$ for a function that maps $x$ to $v$
  and agrees with $f$ for other variables.
  This is rather standard notation, but note that it coincides
  with nameless operation $f[\mapsto v]$ defined in the previous section.
\end{itemize}

%%%%%%%%%%%%%%%%%%%%%%%%%%%%%%%%%%%%%%%%%%%%%%%%%%%%%%%%%%%%%%%%%%%%%%%%%%%%%%%
\section{Further Reading}
