\chapter{Parametric Polymorphism}
A lot of useful lambda-abstractions can assume many different types.
The term $\lambda x . x$ provably has types $\top \rightarrow \top$, 
$(\top + \top) \rightarrow (\top + \top)$, $(\texttt{bool} \times \texttt{int}) \rightarrow (\texttt{bool} \times \texttt{int})$,
etc. Indeed, $\lambda x. x$ has type $\tau \rightarrow \tau$ for any $\tau \in \syncatset{Type}$.
Similarly,
\begin{itemize}
\item $\lambda f. \lambda x. f (f x)$ has type $(\tau \rightarrow \tau) \rightarrow \tau \rightarrow \tau$ for all $\tau \in \syncatset{Type}$,
\item $\lambda f. \lambda g. \lambda x. f (g x)$ has type 
$(\tau_1 \rightarrow \tau_2) \rightarrow (\tau_3 \rightarrow \tau_1) \rightarrow \tau_3 \rightarrow \tau_2$ 
for all $\tau_1$, $\tau_2$, $\tau_3$~$\in \syncatset{Type}$,
\item $\lambda f. \lambda x. \lambda y. f y x$ has type 
$(\tau_1 \rightarrow \tau_2 \rightarrow \tau_3) \rightarrow \tau_2 \rightarrow \tau_1 \rightarrow \tau_3$ 
for all $\tau_1$, $\tau_2$, $\tau_3$~$\in \syncatset{Type}$.
\end{itemize}
When placed in a specific context, these expressions will be assigned their
appropriate concrete types by the simple typing relation. However, let us not
forget how much programmers like to name things. It would be really convenient
to write a let-expression
\[ \texttt{let flip} = \lambda f. \lambda x. \lambda y. f y x \texttt{ in } \ldots \]
and then use \texttt{flip} on inputs of various types further in the program. 
Remembering that let-expressions are desugared into applications, we notice that 
in the type derivation of
\[ (\lambda \texttt{flip}. \ldots) (\lambda f. \lambda x. \lambda y. f y x) \]
the right hand side's type will be derived only once! Perhaps the simplest
example illustrating this problem is as follows: the expression $(\lambda x . x)(\lambda x . x)$
is well-typed, but $(\lambda \texttt{id}. \texttt{id id}) (\lambda x . x)$ is not,
as it requires deriving the type of a self-application.

We now introduce a calculus called System F, the goal of which is to incorporate
statements like ``\texttt{id} has type $\tau \rightarrow \tau$ for all $\tau$'' 
into the type system.
%%%%%%%%%%%%%%%%%%%%%%%%%%%%%%%%%%%%%%%%%%%%%%%%%%%%%%%%%%%%%%%%%%%%%%%%%%%%%%%
\section{System F}

\begin{alignat*}{2}
  \syncatset{Type} & \ni \tau && \Coloneqq
    \tau \to \tau \mid \forall\alpha.\tau \mid \alpha \tag{types}\\
  \syncatset{Expr} & \ni e && \Coloneqq
    v \mid e\;e \mid e\;* \tag{expressions} \\
  \syncatset{Value} & \ni v && \Coloneqq
    \lambda x.e \mid x \mid \Lambda e \tag{values}
\end{alignat*}
The grammar of types is extended with the universal quantifier and type variables,
which are stored in a new environment $\Delta$. Types that appear in the usual
environment $\Gamma$ may contain type variables from $\Delta$. The grammar of expressions
now includes an instantiation operator. Another addition is the value-forming
capital lambda.

According to the typing rules below, capital lambda precedes expressions
whose type is to be generalized. Such a type can in turn be instantiated 
to a desired concrete form by means of the $*$ operator.
\begin{mathpar}
  \inferrule{\Delta,\alpha;\Gamma\vdash e : \tau}
            {\Delta;\Gamma\vdash \Lambda e : \forall\alpha.\tau}

  \inferrule{\Delta;\Gamma\vdash e : \forall\alpha.\tau}
            {\Delta;\Gamma\vdash e\;* : \tau\{\tau'/\alpha\}}
\end{mathpar}

This is not entirely what we are used to in practice. A programmer usually
does not want to keep track of generalizing and instantiating types---especially
not in the very syntax. Our goal was to just write a function and have the type
system understand that it is general enough to be used on various inputs. We
therefore consider the following alternative, implicit typing rules.
\begin{mathpar}
  \inferrule{\Delta,\alpha;\Gamma\vdash e : \tau}
            {\Delta;\Gamma\vdash e : \forall\alpha.\tau}

  \inferrule{\Delta;\Gamma\vdash e : \forall\alpha.\tau}
            {\Delta;\Gamma\vdash e : \tau\{\tau'/\alpha\}}
\end{mathpar}

Unfortunately, with these rules, some problems arise once we add features
like mutable state or \texttt{call/cc} to our language. We can get rid of these problems
by restricting the generalization of types to types of values only. This is known
as the \textit{value restriction}.
\begin{mathpar}
  \inferrule{\Delta,\alpha;\Gamma\vdash v : \tau}
            {\Delta;\Gamma\vdash v : \forall\alpha.\tau}
\end{mathpar}

Returning to the explicit version of System F, it is now easy to check that
$(\lambda \texttt{id}. \texttt{id}*\texttt{id}) (\Lambda \lambda x. x)$ is well-typed.
%%%%%%%%%%%%%%%%%%%%%%%%%%%%%%%%%%%%%%%%%%%%%%%%%%%%%%%%%%%%%%%%%%%%%%%%%%%%%%%
\section{Logical Relations for Polymorphism}
We need to define denotation of universal type.
One possible definition would be:
\[
  \Denot{\forall\alpha\ldotp\tau} = \{ v \mid \forall \tau' \in\syncatset{Type}
    \ldotp v\;*\in \mathcal E \Denot{\Subst\tau{\tau'}x} \}
\]

Despite this definition looking promising, there is a severe problem,
because it is not structural -- type $\Subst\tau{\tau'}x$ could be structurally
bigger than $\tau$. We can try fixing this and add an environment $\Delta$ and
a function $\eta\colon\Delta\to\syncatset{Type}$ to remember type $\tau'$.
This would give us a definition
\begin{alignat*}{2}
  \Denot{\forall\alpha\ldotp\tau}_{\eta} & = \{ v \mid
    \forall \tau' \in \syncatset{Type}.
      v\;* \in \mathcal{E}\Denot{\tau}_{\eta[\alpha \mapsto \tau']} \} \\
  \Denot{\alpha}_{\eta} & = \Denot{\eta(\alpha)}
\end{alignat*}

This is close to proper definition, but it still isn't structural.
Trick that solves this problem is to instead of quantifying over
syntactic types, to do it over semantic types.
Therefore, we arrive at the correct definition, which is
\begin{alignat*}{2}
  \Denot{\forall\alpha\ldotp\tau}_{\eta} & = \{ v \mid
    \forall R \in \semcatset{Type}.
      v\;* \in \mathcal{E}\Denot{\tau}_{\eta[\alpha \mapsto R]} \} \\
  \Denot{\alpha}_{\eta} & = \eta(\alpha)
\end{alignat*}

For all the other denotations, we simply need to carry $\eta$.
For example the denotation of arrow would be defined as

\[
  \Denot{\tau_1 \to \tau_2}_\eta = \{ v \mid
    \forall v' \in \Denot{\tau_1}_\eta\ldotp
      v\;v' \in \mathcal{E}\Denot{\tau_2}_\eta \} \\\\
\]
Since we added a set $\Delta$ which is an environment for type variables,
we need to redefine logical relation from \autoref{sec:log-rel}.
The semantic representation for this type of environment is a function
$\eta\colon\Delta\to\semcatset{Type}$, that for each type variable returns
its meaning in terms of semantic types. Therefore, we have
\[
  \Delta; \Gamma\models e \;\colon\; \tau \iff
    \forall \eta\colon\Delta\to\semcatset{Type}\ldotp
      \forall \gamma\in\Denot\Gamma_\eta\ldotp
        \gamma^*\;e \in\Denot\tau_\eta
\]

%%%%%%%%%%%%%%%%%%%%%%%%%%%%%%%%%%%%%%%%%%%%%%%%%%%%%%%%%%%%%%%%%%%%%%%%%%%%%%%
\section{Further Reading}
